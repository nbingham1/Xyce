% Sandia National Laboratories is a multimission laboratory managed and
% operated by National Technology & Engineering Solutions of Sandia, LLC, a
% wholly owned subsidiary of Honeywell International Inc., for the U.S.
% Department of Energy’s National Nuclear Security Administration under
% contract DE-NA0003525.

% Copyright 2002-2024 National Technology & Engineering Solutions of Sandia,
% LLC (NTESS).

\documentclass[letterpaper]{scrartcl}
\usepackage[hyperindex=true, colorlinks=false]{hyperref}
\usepackage{ltxtable, multirow}
\usepackage{Xyce}
\usepackage{geometry}

\pdfcatalog {/PageMode /UseNone}
\renewcommand{\arraystretch}{1.2}

% Sets the page margins to be the same as the Guides (SAND reports)
\geometry{pdftex, inner=1in, textwidth=6.5in, textheight=9in}

% Gets rid of Section numbers
\setcounter{secnumdepth}{0}

% Set this here once, and use \XyceVersionVar{} in the document
\XyceVersion{7.6}

% ---------------------------------------------------------------------------- %
%
% Set the title, author, and date
%
\title{\XyceTitle{} Parallel Electronic Simulator\\
Version \XyceVersionVar{} Release Notes}

\author{ Sandia National Laboratories}

\date{\today}

% Approved April TBD 2021
% SAND Number SAND2020-TBD

% ---------------------------------------------------------------------------- %
% Start the document

\begin{document}
\maketitle

The \XyceTM{} Parallel Electronic Simulator has been written to support the
simulation needs of Sandia National Laboratories' electrical designers.
\XyceTM{} is a SPICE-compatible simulator with the ability to solve extremely
large circuit problems on large-scale parallel computing platforms, but also
includes support for most popular parallel and serial computers.

For up-to-date information not available at the time these notes were produced,
please visit the \XyceTM{} web page at
{\color{XyceDeepRed}\url{http://xyce.sandia.gov}}.

\tableofcontents
\vspace*{\fill}
\parbox{\textwidth}
{
  \raisebox{0.13in}{\includegraphics[height=0.5in]{snllineblubrd}}
  \hfill
  \includegraphics[width=1.5in]{xyce_flat_white}
}


\newpage
\section{New Features and Enhancements}

\subsubsection*{XDM}
\begin{XyceItemize}

\item XDM now allows models to be redefined within the same scope
        without raising an exception. Previously, the code would exit out
        if a model was redefined. Now, it will just emit a warning.

\item XDM now tranlates the SPECTRE model parameter "VERSION" to "version".

\item HSPICE expressions can be delimited by double quotes. XDM would
    either let this pass through without making any changes (which
    would cause problems with the resultant Xyce netlist), or comment
    it out in some cases (mostly, in expressions in sources). This was
    fixed by adding a HSPICE grammar rule that defines expressions
    delimited by double quotes, and then adding code in the parser
    interface to change the expression delimiters to curly braces.

\end{XyceItemize}

\subsubsection*{New Devices and Device Model Improvements}
\begin{XyceItemize}
\item The voltage-controlled current source (VCCS, or '\texttt{G}'
    device) now supports the \texttt{M} multiplier parameter.
\item The BSIM4 device (level 14 and 54 MOSFET) now supports three
  versions, the 4.6.1 version that has been present for many releases
  and the 4.7.0 and 4.8.2 versions new to this release.  The specific
  version of BSIM4 can be selected by setting the \texttt{VERSION}
  parameter on the \texttt{.MODEL} line for the device.  At this time
  the BSIM4 is the only device in Xyce that supports multiple versions
  in this manner.
\end{XyceItemize}

\subsubsection*{Enhanced Solver Stability, Performance and Features}
\begin{XyceItemize}
\item The new harmonics selection method based on diamond truncation has
  been added for HB analysis.
\item Improvements to the parser have been made to address significant
  slowdowns in processing the open source SkyWater 130nm PDK.  For
  exemplar stress tests, parsing is now up to 20x faster.
\end{XyceItemize}

\subsubsection*{Interface Improvements}
\begin{XyceItemize}
\item The number of warnings output by Xyce can now be controlled by
  the command-line option ``-max-warnings \#''.  The default maximum
  number of warnings is now 100, before it was unlimited.
\end{XyceItemize}

\subsubsection*{Important Announcements}
\begin{XyceItemize}
\item The model interpolation technique described in the \Xyce{}
  Reference Guide in section 2.1.18 has been marked as deprecated, and
  will be removed in a future release of Xyce.
\end{XyceItemize}

\newpage
\section{Defects Fixed in this Release}
% Sandia National Laboratories is a multimission laboratory managed and
% operated by National Technology & Engineering Solutions of Sandia, LLC, a
% wholly owned subsidiary of Honeywell International Inc., for the U.S.
% Department of Energy's National Nuclear Security Administration under
% contract DE-NA0003525.

% Copyright 2002-2024 National Technology & Engineering Solutions of Sandia,
% LLC (NTESS).

% Sandia National Laboratories is a multimission laboratory managed and
% operated by National Technology & Engineering Solutions of Sandia, LLC, a
% wholly owned subsidiary of Honeywell International Inc., for the U.S.
% Department of Energy's National Nuclear Security Administration under
% contract DE-NA0003525.

% Copyright 2002-2024 National Technology & Engineering Solutions of Sandia,
% LLC (NTESS).


%%
%% Fixed Defects.
%%
{
\small

\begin{longtable}[h] {>{\raggedright\small}m{2in}|>{\raggedright\let\\\tabularnewline\small}m{3.5in}}
     \caption{Fixed Defects.  The Xyce team has multiple issue
     trackers, and the table below indicates fixed issues by
     indentifying both the tracker and the issue number.  Further,
     some issues are reported by open source users on GitHub and these
     issues may be tracked using multiple issue numbers.} \\ \hline
     \rowcolor{XyceDarkBlue} \color{white}\textbf{Defect} & \color{white}\textbf{Description} \\ \hline
     \endfirsthead
     \caption[]{Fixed Defects.  Note that we have two multiple issue tracking systems for Sandia Users.
     SON and SRN refer to our legacy open- and restricted-network Bugzilla system, and Gitlab refers to issues in our gitlab repositories.  } \\ \hline
     \rowcolor{XyceDarkBlue} \color{white}\textbf{Defect} & \color{white}\textbf{Description} \\ \hline
     \endhead

\textbf{Xyce Backlog Bugs/1}: Mutual inductor jacobian error &
A jacobian term in the nonlinear mutual inductor was wrong when the
bias was changing across the primary inductor.  This jacobian error
could lead to slower nonlinear convergence in transient simulations.
It has been fixed. \\ \hline

\textbf{Xyce Backlog Bugs/2}: Mutual inductors have a bad device connectivity map &
The device connectivity map is used to determine the path to ground
for error checking.  The linear and nonlinear mutual inductors were
not correctly setting up the device connectivity map and this resulted
in false warnings that some circuit nodes did not have a path to
ground.  The warning was invalid and this issue did not effect Xyce's
calculations.  But the warning was incorrect and could cause
confusion.  The code has been fixed and this warning should no longer
appear under false conditions. \\ \hline

\textbf{Xyce Backlog Bugs/7}: Infinite loop results when ".ends" is missing from netlist &
If a netlist ended without closing a subcircuit with ".ends" and
".end" was not used to signify the end of the netlist, then Xyce would
either execute an infinite loop or seg fault.  The parser now
identifies this error and cleanly exits the simulation with an
informative message. \\ \hline

\textbf{Xyce Backlog Bugs/8}: MOSFET multiplicity not applied to noise &
None of the legacy MOSFETs (level 1,2,3, 6, 9(BSIM3) and 14 (BSIM4))
in Xyce were correctly applying the multiplicity factor (``M='') to
noise terms.  They do now. \\ \hline

\textbf{Xyce Backlog Bugs/19}: Error in the failure history output by Xyce in
parallel &  A runtime error was observed in the failure history output when
the node name string was not available on the processor accessing the string.
The character string is now being broadcast from the processor that owns the
character string to all other processors. \\ \hline

\textbf{Xyce Project Backlog/474}: Remove spice-incompatible breakdown parameter algorithm from level 1 and 2 diode &
When a user specifies both \texttt{IBV} and \texttt{BV} to the diode,
an iterative technique is employed to assure that the forward and
reverse regions match, which can adjust the breakdown voltage if
necessary.  The SPICE3F5 algorithm may or may not converge, and can
sometimes produce unreasonable solutions to the matching problem.  In
2007 an improved algorithm was devised and implemented in the \Xyce{}
diode, but this broke strict SPICE compatibility.  This algorithm has
been reverted to the SPICE-compatible method.  Some small differences
in simulation results as compared to prior versions of \Xyce{} may
result from this change, but the changes will make the results more
compatible with those from other simulators.  \\ \hline

\textbf{Xyce Project Backlog/471}: Removal of spice-incompatible IRF parameter from diode &
The \Xyce{} diode has, since 2005, supported a Xyce-specific, SPICE3F5
incompatible parameter called \texttt{IRF} that purported to allow a
user to fudge the reverse current expression to match experimental
data.  Though the original implementation was intended to fall back to
strict SPICE3F5 compliance when \texttt{IRF} was not specified, later
development broke it and led to non-physical, temperature-dependent
discontinuities in the diode formulation even when \texttt{IRF} was
not specified.  In \Xyce{} 6.11 we fixed the fall-back behavior, but
marked the parameter as deprecated and had \Xyce{} report a warning if
it was specified.  As of \Xyce{} 7.6 this feature has been completely
removed, restoring strict SPICE3F5 compatibility. \\ \hline

\textbf{Xyce Project Backlog/260}: Fix version string in CMake development builds &
Xyce reports its version using \texttt{Xyce -v}.  With development versions of
Xyce, the reported version should include the Git SHA against which the code
was compiled along with the time of compilation. Prior to this change, CMake
would include the SHA and time at \emph{configure} time, not compilation time.
That is now fixed. \\ \hline

\textbf{Xyce Project Backlog/285}: CMake support for XyceCInterface &
The C-interface to the C++ object N\_CIR\_Xyce did not have the needed
files for building under CMake.  This issue has been resolved and the 
C-interface is now built and installed as part of Xyce. \\ \hline

\textbf{Xyce Project Backlog/302}: Xyce legacy MOSFETs do not recognize VT0 &
Xyce did not properly recognize that VT0 should be an alias for VTO in
the levels 1 through 6 legacy MOSFET devices.  These have been
recognized by every SPICE derivative since SPICE3 (though were not
recognized by SPICE2), and are now recognized by Xyce, too. \\ \hline

\textbf{Xyce Project Backlog/151}: Xyce reports error "Directory node not found" &
Xyce reports this error when non-existent solution nodes being referred to
by circuit devices and cannot be resolved.  The simulator now provides a more
informative error that indicates which solution node has not been resolved and
what device refers to this node. \\ \hline

\textbf{Xyce Project Backlog/419, 420, 352}:  Xyce only supports version 4.6.1 of the BSIM4 &
Until now, Xyce only supported version 4.6.1 of the BSIM4 and ignored
any setting of the \texttt{VERSION} parameter in the model card for level
14 or 54.  As of this release, Xyce supports multiple versions (4.6.1,
4.7.0, and 4.8.2), all specified as level 14 or 54 and selected by
the \texttt{VERSION} parameter in the model card.  The BSIM4 is
currently the only device in Xyce that supports multiple versions in
this manner. \\ \hline


\textbf{Xyce Project Backlog/448}:  Auger recombination function used by
the TCAD (PDE) devices is scaled incorrectly & The Auger recombination
term used by the TCAD devices in Xyce was scaled incorrectly, and this
resulted in that term being nearly zero.  This has been corrected.
\\ \hline

\textbf{Xyce Project Backlog/455}:  Xyce gives a bad error message for
malformed voltage source & Under some circumstances, a grammar mistake
in the netlist specification of an independent source caused Xyce to
exit with a fatal error, but with an incomprehensible error message.
This has been corrected.
\\ \hline

\textbf{Xyce Project Backlog/bugs/23}:
Fix the Philips mobility model for the TCAD devices & The Philips
mobility model had a bug in it that resulted in NaNs being produced
during an intermediate calculation.  The model would discard these
NaNs under most circumstances.  However, they were computed inside of
a while loop, and on some compilers these NaNs prevented the while
loop from exiting.  This model has been completely rewritten and this
behavior will no longer occur.
\\ \hline

\textbf{Xyce Project Backlog/264}:  Modify Xyce CMake to use the modern CMake in Trilinos &
Trilinos 13.5 introduces support for modern CMake targets. As part of
the transition to CMake targets, several Trilinos defined CMake
variables have been marked as deprecated (including some variables
containing library names, library paths, and include
directories). When building Xyce from source with CMake, dependencies
in Trilinos are now referenced through Trilinos provided CMake targets
(instead of explicitly using library names and paths). These CMake
build system changes are not backwards compatible; the CMake scripts
require Trilinos 13.5 or greater to build Xyce.
\\ \hline

\end{longtable}
}


\newpage
\section{Supported Platforms}
\subsection*{Certified Support}
The following platforms have been subject to certification testing for the
\Xyce{} version 7.6 release.
\begin{XyceItemize}
  \item Red Hat Enterprise Linux${}^{\mbox{\textregistered}}$ 7, x86-64 (serial and parallel)
  \item Microsoft Windows 10${}^{\mbox{\textregistered}}$, x86-64 (serial)
  \item Apple${}^{\mbox{\textregistered}}$ macOS, x86-64 (serial and parallel)
\end{XyceItemize}


\subsection*{Build Support}
Though not certified platforms, \Xyce{} has been known to run on the following
systems.
\begin{XyceItemize}
  \item FreeBSD 12.X on Intel x86-64 and AMD64 architectures (serial
    and parallel)
  \item Distributions of Linux other than Red Hat Enterprise Linux 6
  \item Microsoft Windows under Cygwin and MinGW
\end{XyceItemize}


\section{\Xyce{} Release \XyceVersionVar{} Documentation}
The following \Xyce{} documentation is available on the \Xyce{} website in pdf
form.
\begin{XyceItemize}
  \item \Xyce{} Version \XyceVersionVar{} Release Notes (this document)
  \item \Xyce{} Users' Guide, Version \XyceVersionVar{}
  \item \Xyce{} Reference Guide, Version \XyceVersionVar{}
  \item \Xyce{} Mathematical Formulation
  \item Power Grid Modeling with \Xyce{}
  \item Application Note: Coupled Simulation with the \Xyce{} General
    External Interface
  \item Application Note: Mixed Signal Simulation with \Xyce{} 7.2
\end{XyceItemize}
Also included at the \Xyce{} website as web pages are the following.
\begin{XyceItemize}
  \item Frequently Asked Questions
  \item Building Guide (instructions for building \Xyce{} from the source code)
  \item Running the \Xyce{} Regression Test Suite
  \item \Xyce{}/ADMS Users' Guide
  \item Tutorial:  Adding a new compact model to \Xyce{}
\end{XyceItemize}


\section{External User Resources}
\begin{itemize}
  \item Website: {\color{XyceDeepRed}\url{http://xyce.sandia.gov}}
  \item Google Groups discussion forum:
    {\color{XyceDeepRed}\url{https://groups.google.com/forum/#!forum/xyce-users}}
  \item Email support:
    {\color{XyceDeepRed}\href{mailto:xyce@sandia.gov}{xyce@sandia.gov}}
  \item Address:
    \begin{quote}
            Electrical Models and Simulation Dept.\\
            Sandia National Laboratories\\
            P.O. Box 5800, M.S. 1177\\
            Albuquerque, NM 87185-1177 \\
    \end{quote}
\end{itemize}

\vspace*{\fill}
\noindent
Sandia National Laboratories is a multimission laboratory managed and
operated by National Technology and Engineering Solutions of Sandia,
LLC, a wholly owned subsidiary of Honeywell International, Inc., for
the U.S. Department of Energy's National Nuclear Security
Administration under contract DE-NA0003525.

SAND2022-14883 O
\end{document}

