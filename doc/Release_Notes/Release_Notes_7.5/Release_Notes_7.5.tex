% Sandia National Laboratories is a multimission laboratory managed and
% operated by National Technology & Engineering Solutions of Sandia, LLC, a
% wholly owned subsidiary of Honeywell International Inc., for the U.S.
% Department of Energy’s National Nuclear Security Administration under
% contract DE-NA0003525.

% Copyright 2002-2024 National Technology & Engineering Solutions of Sandia,
% LLC (NTESS).

\documentclass[letterpaper]{scrartcl}
\usepackage[hyperindex=true, colorlinks=false]{hyperref}
\usepackage{ltxtable, multirow}
\usepackage{Xyce}
\usepackage{geometry}

\pdfcatalog {/PageMode /UseNone}
\renewcommand{\arraystretch}{1.2}

% Sets the page margins to be the same as the Guides (SAND reports)
\geometry{pdftex, inner=1in, textwidth=6.5in, textheight=9in}

% Gets rid of Section numbers
\setcounter{secnumdepth}{0}

% Set this here once, and use \XyceVersionVar{} in the document
\XyceVersion{7.5}

% ---------------------------------------------------------------------------- %
%
% Set the title, author, and date
%
\title{\XyceTitle{} Parallel Electronic Simulator\\
Version \XyceVersionVar{} Release Notes}

\author{ Sandia National Laboratories}

\date{\today}

% Approved April TBD 2021
% SAND Number SAND2020-TBD

% ---------------------------------------------------------------------------- %
% Start the document

\begin{document}
\maketitle

The \XyceTM{} Parallel Electronic Simulator has been written to support the
simulation needs of Sandia National Laboratories' electrical designers.
\XyceTM{} is a SPICE-compatible simulator with the ability to solve extremely
large circuit problems on large-scale parallel computing platforms, but also
includes support for most popular parallel and serial computers.

For up-to-date information not available at the time these notes were produced,
please visit the \XyceTM{} web page at
{\color{XyceDeepRed}\url{http://xyce.sandia.gov}}.

\tableofcontents
\vspace*{\fill}
\parbox{\textwidth}
{
  \raisebox{0.13in}{\includegraphics[height=0.5in]{snllineblubrd}}
  \hfill
  \includegraphics[width=1.5in]{xyce_flat_white}
}


\newpage
\section{New Features and Enhancements}

\subsubsection*{XDM}
\begin{XyceItemize}
    \item Translations of netlists using binned models will no longer
      have the \texttt{.OPTION PARSER MODEL\_BINNING=TRUE}
      statement. This option is turned on by default in \Xyce{} now
      and therefore no longer needs to be explicitly declared.
    \item Duplicate output variables will now be removed from the
      \texttt{.PRINT} line.
    \item The expression and function pre-processing capabilities have
      been removed from XDM.
    \item [spectre] The \texttt{else} conditional block in
      \texttt{if-else} statments will now be commented out by XDM,
      since conditional statement support is not fully mature yet in
      \Xyce{}.
    \item [pspice] Fixed bug where instance parameters for temperature
      coefficients for resistors were not translated.
\end{XyceItemize}

\subsubsection*{New Devices and Device Model Improvements}
\begin{XyceItemize}
  \item The L-UTSOI MOSFET model (level 10240 MOSFET) is now available
    under an open source license and the source code is now part of
    the main Xyce source code repository.  It had previously only been
    available in our ``non free'' binaries (``XyceNF'').
  \item The level 1 BJT now supports a multiplicity factor \texttt{M=}.
\end{XyceItemize}

\subsubsection*{Enhanced Solver Stability, Performance and Features}
\begin{XyceItemize}
\item Source stepping, which is one of the solver techniques that
  \Xyce{} automatically attempts when trying to solve the DCOP, now
  includes current sources as well as voltage sources.  As a result
  circuits using current sources are much more robust for the DCOP
  calculation.
\item The new harmonics selection method based on box truncation has
  been added for HB analysis
\item Parameter handling for \texttt{.SAMPLING} and other UQ methods
  is now much more efficient (see issue 306 in the fixed defects
  table).
\end{XyceItemize}

\subsubsection*{Interface Improvements}
\begin{XyceItemize}
  \item TRIG-TARG measures are now supported for .AC, .DC and .NOISE
    analyses.  In addition, the TRIG-TARG measure is now more
    compatible with both ngspice and HSPICE.  See gitlab-ex issues 220
    and 289 in the Fixed Defects Table for more details.
  \item ``Continuous mode'' measures that may return more than one
    result are now supported for the TRIG-TARG measure type for .AC,
    .DC, .NOISE and .TRAN analyses.
  \item The scale parameter, which is used to automatically scale
    device sizing parameters is supported and can be specified using
    either \texttt{.option scale=number} or \texttt{.options device
      scale=number}.  The new syntax is
   compatible with most other SPICE-style simulators.
  \item Added the mixed signal interface function
    \texttt{Simulator::getTimeVoltagePairsSz} supplies the maximun
    number of time, voltage or state values for an ADC in a subsequent
    call to \texttt{Simulator:: getTimeVoltagePairs} or
    \texttt{Simulator::getTimeStatePairs}.
  \item Added the mixed signal interface functions
    \texttt{xyce\_getTimeVoltagePairs\-ADCLimitData()} and
    \texttt{xyce\-\_getTimeStatePairs\-ADCLimitData()} to allow the
    calling program to specify the maximum size of the allocated
    memory that can be used for copying ADC time-voltage and
    time-state history data.  This allows the caller to allocate only
    the required amount of memory and ensure that Xyce cannot
    overwrite the memory area.  The existing c-interface functions,
    \texttt{xyce\_getTimeVoltagePairs()} and
    \texttt{xyce\_getTimeStatePairs()} are now deprecated. They work
    in this version of \Xyce{} as they did in the past but they will
    be removed from a future version of \Xyce{}.  See the Application
    note, Mixed Signal Simulation with \Xyce{} 7.5 for further
    details.
  \item The Xyce Python interface file \texttt{xyce\-interface.py} is
    now compatible with Python 3.x.  The interface file is still
    backwards compatible with Python 2.7. The Mixed Signal Simulation
    with \Xyce{} 7.5 Application Note has examples of using Xyce from
    Python.
  \item Implicit multipliers on subcircuits (\texttt{M=}) are now supported.
  \item When a parameter of a given name is defined multiple times in
    the netlist, \Xyce{} now has several options for how this is
    handled.  Before, \Xyce{} would silently allow multiply-defined
    parameters, and use the last one encountered during parsing.  There
    is now a command line option (\texttt{-redefined\_params}), which
    can be used to make this a fatal error, or to use the first, rather
    than last multiply-defined parameter.
  \item Subcircuit instance parameters are now allowed to reference 
    other subcircuit instance parameters on the same line.
  \item When any measure fails, Xyce now reports ``FAILED'' in both the
    console output and ``.mt\#'' files by default.  This new behavior is
    the same as what Xyce 7.4 would have done with the ``.options
    MEASURE MEASFAIL=1'' option specified.  Previous versions of Xyce
    would have output ``-1'' for the failed measure in the ``.mt\#''
    file by default.  The old behavior may be forced by specifying
    ``.options MEASURE MEASFAIL=0 DEFAULT\_VAL=-1'' if desired.
\end{XyceItemize}

\subsubsection*{Xyce/ADMS Improvements}
\begin{XyceItemize}
  \item Several bugs were addressed and are listed in the fixed defects table.
\end{XyceItemize}

\subsubsection*{Important Announcements}
\begin{XyceItemize}
\item The model interpolation technique described in the \Xyce{}
  Reference Guide in section 2.1.18 has been marked as deprecated, and
  will be removed in a future release of Xyce.
\item \Xyce{} binary installers now contain all the files that would
  be installed by ``make install'' instead of just the Xyce binary.
  This includes Xyce headers and library files that would be used to
  link external codes to Xyce.  Use of the \Xyce{} executable itself
  is unchanged by this packaging update.  Use of these headers and
  libraries in user code requires that the user have the same
  compilers that the Xyce team used to build the binary, which may not
  be the case for all users.
\end{XyceItemize}

\newpage
\section{Interface Changes in this Release}
% Sandia National Laboratories is a multimission laboratory managed and
% operated by National Technology & Engineering Solutions of Sandia, LLC, a
% wholly owned subsidiary of Honeywell International Inc., for the U.S.
% Department of Energy’s National Nuclear Security Administration under
% contract DE-NA0003525.

% Copyright 2002-2024 National Technology & Engineering Solutions of Sandia,
% LLC (NTESS).

% Sandia National Laboratories is a multimission laboratory managed and
% operated by National Technology & Engineering Solutions of Sandia, LLC, a
% wholly owned subsidiary of Honeywell International Inc., for the U.S.
% Department of Energy’s National Nuclear Security Administration under
% contract DE-NA0003525.

% Copyright 2002-2024 National Technology & Engineering Solutions of Sandia,
% LLC (NTESS).


%%
%% Changes to Xyce input since the last release.
%%
{
\small

\begin{longtable}[h] {>{\raggedright\small}m{2in}|>{\raggedright\let\\\tabularnewline\small}m{3.5in}}
  \caption{Changes to netlist specification since the last release.\label{newUsage}} \\ \hline
  \rowcolor{XyceDarkBlue}
  \color{white}\textbf{Change} &
  \color{white}\textbf{Detail} \\ \hline \endfirsthead
  \caption[]{Changes to netlist specification since the last release.\label{newUsage}} \\ \hline
  \rowcolor{XyceDarkBlue}
  \color{white}\textbf{Change} &
  \color{white}\textbf{Detail} \\ \hline \endhead

Several unused nonlinear solver options removed &
NORMLVL, DLSDEBUG, LINOPT, MEMORY, and constraint
backtracking parameters (CONSTRAINTBT/ CONSTRAINTMAX/ CONSTRAINTMIN/ CONSTRAINTCHANGE)
have been removed from the set of nonlinear solver options, since they are effectively unused.  \\ \hline

\end{longtable}
}


\newpage
\section{Defects Fixed in this Release}
% Sandia National Laboratories is a multimission laboratory managed and
% operated by National Technology & Engineering Solutions of Sandia, LLC, a
% wholly owned subsidiary of Honeywell International Inc., for the U.S.
% Department of Energy's National Nuclear Security Administration under
% contract DE-NA0003525.

% Copyright 2002-2024 National Technology & Engineering Solutions of Sandia,
% LLC (NTESS).

% Sandia National Laboratories is a multimission laboratory managed and
% operated by National Technology & Engineering Solutions of Sandia, LLC, a
% wholly owned subsidiary of Honeywell International Inc., for the U.S.
% Department of Energy's National Nuclear Security Administration under
% contract DE-NA0003525.

% Copyright 2002-2024 National Technology & Engineering Solutions of Sandia,
% LLC (NTESS).


%%
%% Fixed Defects.
%%
{
\small

\begin{longtable}[h] {>{\raggedright\small}m{2in}|>{\raggedright\let\\\tabularnewline\small}m{3.5in}}
     \caption{Fixed Defects.  The Xyce team has multiple issue
     trackers, and the table below indicates fixed issues by
     indentifying both the tracker and the issue number.  Further,
     some issues are reported by open source users on GitHub and these
     issues may be tracked using multiple issue numbers.} \\ \hline
     \rowcolor{XyceDarkBlue} \color{white}\textbf{Defect} & \color{white}\textbf{Description} \\ \hline
     \endfirsthead
     \caption[]{Fixed Defects.  Note that we have two multiple issue tracking systems for Sandia Users.
     SON and SRN refer to our legacy open- and restricted-network Bugzilla system, and Gitlab refers to issues in our gitlab repositories.  } \\ \hline
     \rowcolor{XyceDarkBlue} \color{white}\textbf{Defect} & \color{white}\textbf{Description} \\ \hline
     \endhead

\textbf{Xyce Backlog Bugs/1}: Mutual inductor jacobian error &
A jacobian term in the nonlinear mutual inductor was wrong when the
bias was changing across the primary inductor.  This jacobian error
could lead to slower nonlinear convergence in transient simulations.
It has been fixed. \\ \hline

\textbf{Xyce Backlog Bugs/2}: Mutual inductors have a bad device connectivity map &
The device connectivity map is used to determine the path to ground
for error checking.  The linear and nonlinear mutual inductors were
not correctly setting up the device connectivity map and this resulted
in false warnings that some circuit nodes did not have a path to
ground.  The warning was invalid and this issue did not effect Xyce's
calculations.  But the warning was incorrect and could cause
confusion.  The code has been fixed and this warning should no longer
appear under false conditions. \\ \hline

\textbf{Xyce Backlog Bugs/7}: Infinite loop results when ".ends" is missing from netlist &
If a netlist ended without closing a subcircuit with ".ends" and
".end" was not used to signify the end of the netlist, then Xyce would
either execute an infinite loop or seg fault.  The parser now
identifies this error and cleanly exits the simulation with an
informative message. \\ \hline

\textbf{Xyce Backlog Bugs/8}: MOSFET multiplicity not applied to noise &
None of the legacy MOSFETs (level 1,2,3, 6, 9(BSIM3) and 14 (BSIM4))
in Xyce were correctly applying the multiplicity factor (``M='') to
noise terms.  They do now. \\ \hline

\textbf{Xyce Backlog Bugs/19}: Error in the failure history output by Xyce in
parallel &  A runtime error was observed in the failure history output when
the node name string was not available on the processor accessing the string.
The character string is now being broadcast from the processor that owns the
character string to all other processors. \\ \hline

\textbf{Xyce Project Backlog/474}: Remove spice-incompatible breakdown parameter algorithm from level 1 and 2 diode &
When a user specifies both \texttt{IBV} and \texttt{BV} to the diode,
an iterative technique is employed to assure that the forward and
reverse regions match, which can adjust the breakdown voltage if
necessary.  The SPICE3F5 algorithm may or may not converge, and can
sometimes produce unreasonable solutions to the matching problem.  In
2007 an improved algorithm was devised and implemented in the \Xyce{}
diode, but this broke strict SPICE compatibility.  This algorithm has
been reverted to the SPICE-compatible method.  Some small differences
in simulation results as compared to prior versions of \Xyce{} may
result from this change, but the changes will make the results more
compatible with those from other simulators.  \\ \hline

\textbf{Xyce Project Backlog/471}: Removal of spice-incompatible IRF parameter from diode &
The \Xyce{} diode has, since 2005, supported a Xyce-specific, SPICE3F5
incompatible parameter called \texttt{IRF} that purported to allow a
user to fudge the reverse current expression to match experimental
data.  Though the original implementation was intended to fall back to
strict SPICE3F5 compliance when \texttt{IRF} was not specified, later
development broke it and led to non-physical, temperature-dependent
discontinuities in the diode formulation even when \texttt{IRF} was
not specified.  In \Xyce{} 6.11 we fixed the fall-back behavior, but
marked the parameter as deprecated and had \Xyce{} report a warning if
it was specified.  As of \Xyce{} 7.6 this feature has been completely
removed, restoring strict SPICE3F5 compatibility. \\ \hline

\textbf{Xyce Project Backlog/260}: Fix version string in CMake development builds &
Xyce reports its version using \texttt{Xyce -v}.  With development versions of
Xyce, the reported version should include the Git SHA against which the code
was compiled along with the time of compilation. Prior to this change, CMake
would include the SHA and time at \emph{configure} time, not compilation time.
That is now fixed. \\ \hline

\textbf{Xyce Project Backlog/285}: CMake support for XyceCInterface &
The C-interface to the C++ object N\_CIR\_Xyce did not have the needed
files for building under CMake.  This issue has been resolved and the 
C-interface is now built and installed as part of Xyce. \\ \hline

\textbf{Xyce Project Backlog/302}: Xyce legacy MOSFETs do not recognize VT0 &
Xyce did not properly recognize that VT0 should be an alias for VTO in
the levels 1 through 6 legacy MOSFET devices.  These have been
recognized by every SPICE derivative since SPICE3 (though were not
recognized by SPICE2), and are now recognized by Xyce, too. \\ \hline

\textbf{Xyce Project Backlog/151}: Xyce reports error "Directory node not found" &
Xyce reports this error when non-existent solution nodes being referred to
by circuit devices and cannot be resolved.  The simulator now provides a more
informative error that indicates which solution node has not been resolved and
what device refers to this node. \\ \hline

\textbf{Xyce Project Backlog/419, 420, 352}:  Xyce only supports version 4.6.1 of the BSIM4 &
Until now, Xyce only supported version 4.6.1 of the BSIM4 and ignored
any setting of the \texttt{VERSION} parameter in the model card for level
14 or 54.  As of this release, Xyce supports multiple versions (4.6.1,
4.7.0, and 4.8.2), all specified as level 14 or 54 and selected by
the \texttt{VERSION} parameter in the model card.  The BSIM4 is
currently the only device in Xyce that supports multiple versions in
this manner. \\ \hline


\textbf{Xyce Project Backlog/448}:  Auger recombination function used by
the TCAD (PDE) devices is scaled incorrectly & The Auger recombination
term used by the TCAD devices in Xyce was scaled incorrectly, and this
resulted in that term being nearly zero.  This has been corrected.
\\ \hline

\textbf{Xyce Project Backlog/455}:  Xyce gives a bad error message for
malformed voltage source & Under some circumstances, a grammar mistake
in the netlist specification of an independent source caused Xyce to
exit with a fatal error, but with an incomprehensible error message.
This has been corrected.
\\ \hline

\textbf{Xyce Project Backlog/bugs/23}:
Fix the Philips mobility model for the TCAD devices & The Philips
mobility model had a bug in it that resulted in NaNs being produced
during an intermediate calculation.  The model would discard these
NaNs under most circumstances.  However, they were computed inside of
a while loop, and on some compilers these NaNs prevented the while
loop from exiting.  This model has been completely rewritten and this
behavior will no longer occur.
\\ \hline

\textbf{Xyce Project Backlog/264}:  Modify Xyce CMake to use the modern CMake in Trilinos &
Trilinos 13.5 introduces support for modern CMake targets. As part of
the transition to CMake targets, several Trilinos defined CMake
variables have been marked as deprecated (including some variables
containing library names, library paths, and include
directories). When building Xyce from source with CMake, dependencies
in Trilinos are now referenced through Trilinos provided CMake targets
(instead of explicitly using library names and paths). These CMake
build system changes are not backwards compatible; the CMake scripts
require Trilinos 13.5 or greater to build Xyce.
\\ \hline

\end{longtable}
}


\newpage
\section{Known Defects and Workarounds}
% Sandia National Laboratories is a multimission laboratory managed and
% operated by National Technology & Engineering Solutions of Sandia, LLC, a
% wholly owned subsidiary of Honeywell International Inc., for the U.S.
% Department of Energy’s National Nuclear Security Administration under
% contract DE-NA0003525.

% Copyright 2002-2024 National Technology & Engineering Solutions of Sandia,
% LLC (NTESS).

% Sandia National Laboratories is a multimission laboratory managed and
% operated by National Technology & Engineering Solutions of Sandia, LLC, a
% wholly owned subsidiary of Honeywell International Inc., for the U.S.
% Department of Energy’s National Nuclear Security Administration under
% contract DE-NA0003525.

% Copyright 2002-2024 National Technology & Engineering Solutions of Sandia,
% LLC (NTESS).


%%
%% Known defects and workarounds.
%%
%% This section should highlight significant defects that were not fixed in
%% the release

{
\small

\begin{longtable}[h] {>{\raggedright\small}m{2in}|>{\raggedright\let\\\tabularnewline\small}m{3.5in}}
  \caption{Known Defects and Workarounds.} \\ \hline
  \rowcolor{XyceDarkBlue} \color{white}\textbf{Defect} & \color{white}\textbf{Description} \\ \hline \endfirsthead
  \caption[]{Known Defects and Workarounds.} \\ \hline
  \rowcolor{XyceDarkBlue} \color{white}\textbf{Defect} & \color{white}\textbf{Description}
  \\ \hline \endhead
% EXAMPLE:
%\textbf{bug number-SRN}: bug title & Description of KNOWN BUG THAT HAS NOT BEEN
  %FIXED.

%\textbf{\textit{Workaround}}: Describe how to work around this bug.
%\\ \hline
%
%
\textbf{1262-SON}: Duplicate L device definitions are not a parsing error
when one of the duplicate L devices is part of a K device &
As an example, this netlist will not produce a parsing error.  Instead,
the first L1 definition will be used in the K1 device definition.
\begin{verbatim}
* parsing fails to detect duplicate L1 devices
V1 1 0 SIN(0 1 1KHz)
L1 1 2 1e-3
R1 2 0 1
C1 2 0 1e-9
* mutual inductor definition, with duplicate L1 device
L1 4 0 1e-6
L2 3 0 1mH
K1 L1 L2 0.75
.TRAN 0 1ms
.PRINT TRAN V(1) v(2)
.END
\end{verbatim}

\textbf{\textit{Workaround}}:
There is none.
\\ \hline

\textbf{1241-SON}: Expression library parsing bottleneck on large expressions &
It has been determined that the expression library in Xyce can be the
source of a severe parsing bottleneck when expressions are large and
complex.  Expressions of this sort show up most often when parsing
large PDKs with complex use of the \texttt{.FUNC} feature, and when
using the ``tablefile'' feature to import a large file of time/voltage
pairs for use in a \texttt{B} source.

\textbf{\textit{Workaround}}:
There is currently no workaround for the issue of complex PDK function
use, and the team is working on fixing this issue by redesigning the
way Xyce handles expressions with user defined functions.  For the
``tablefile'' issue, one should avoid using \texttt{B} sources with
``tablefile'' to read in large tables, and instead use the ``PWL
FILE'' option of the \texttt{V} source, which does not have this
parsing issue.
\\ \hline

\textbf{1085-SON}: Expression library mishandles \texttt{.FUNC} definitions of functions that begin with ``I'' and are two characters long &
\Xyce{}'s expression library assumes that all terms of the form
``\texttt{Ix(<arguments>)}'' are lead current expressions, where ``x'' is
either a lead designator such as ``D'', ``G'' , or ''S'' for a MOSFET
or ``C'', ``B'', ``E'' for a BJT, or a digit indicating the pin number
of the device associated with the lead.  This assumption makes it
impossible for users to define a function with a two-character name
starts with ``I''.  Unfortunately, the parser does not warn of this
problem should a user define such a function, and the first indication
of something being wrong is an unhelpful error message about an
``undefined parameter or function'' where the problematic function is
used.

\textbf{\textit{Workaround}}: Do not use function names of two character
length that begin with the letter ``I''.  If you are making use of a
vendor-supplied library that includes definitions of functions such as
``I0'', you will have to modify the library to change the function
name and all the instances of its use.
\\ \hline

\textbf{1037-SON}: The use of non-constant values in .PARAM
statements may lead to unexpected results &  This netlist line
(\texttt{.PARAM PA = \{TEMP\}}) is forbidden in Xyce since
the special variable \texttt{TEMP} is not constant.  However,
that netlist line will not produce a \Xyce{} parsing error, and
the value of \texttt{PA} in the simulation may then be set to 
zero in some contexts.

\textbf{\textit{Workaround}}: Non-constant values should only
be used in \texttt{.GLOBAL PARAM} statements in \Xyce{}. This
restriction may be different than in other Spice-like 
simulators.
\\ \hline

\textbf{1031-SON}: .OP output is incomplete in parallel & When \Xyce{}
is run in parallel, the \texttt{.OP} output may be incomplete. 

\textbf{\textit{Workaround}}: One workaround is to run the netlist in 
serial.  Another one is to use these command line options: \texttt{
-per-processor -l output}.  In that case, the per-processor log files 
will have the \texttt{.OP} information for the devices that were 
instantiated on each processor.
\\ \hline 

  \textbf{1009-SON}:  Transient adjoint sensitivities don't work with \texttt{.STEP}
  & Transient adjoint sensitivities require backward integrations to be performed after the primary transient forward integration.  Doing this properly requires information to be stored during the forward solve, and for certain bookkeeping to be performed.  Currently, these extra operations to support transient adjoints are not properly set up for \texttt{.STEP} analysis.

\textbf{\textit{Workaround}}: None
\\ \hline

  \textbf{1006-SON}:   SDT (expression library time integration) derivatives are not supported, so SDT can't be used for sensitivity analysis objective functions &
  SDT is a function supported by the \Xyce{} expression library to compute numerical time integration.  When this function is used, the expression library does not produce correct derivatives.  This impacts Jacobian matrix entries, when SDT is used with a Bsrc, and it also impacts sensitivity analysis, when SDT is used in an objective function.  For the former case, this can result in a lack of robustness for circuits that contain SDT-Bsrc devices.  For the latter case, the objective function will simply be incorrect.

\textbf{\textit{Workaround}}: None
\\ \hline

\textbf{1004-SON}: Ill-defined .STEP behavior for "default parameters" for 
transient sources (SIN, EXP, PWL, PULSE and SFFM) & If, for example,
these netlist lines are used in a transient (\texttt{.TRAN}) simulation:
\begin{verbatim}
V1 1 0 SIN(0 1 1)
.STEP V1 1 2 1
\end{verbatim}
then \Xyce{} will run the simulation without warnings or errors, but
no instance parameter of source \texttt{V1} will be stepped.  

\textbf{\textit{Workaround}}: Explicitly use the desired stepped parameter
(e.g., \texttt{V0}) on the \texttt{.STEP} line.  For example, 
\texttt{.STEP V1:V0 1 2 1} would work correctly.
\\ \hline

\textbf{991-SON}: Non-physical BH Loops in non-linear mutual inductor &
Nonlinear mutual inductors that have high coupling coefficients (i.e. 
model parameter \texttt{ALPHA} over 1.0e-4) and low loss characteristics 
(i.e. zero \texttt{GAP}) can produce B-H loops with nonphysical hysteresis.

\textbf{\textit{Workaround}}: Lower \texttt{ALPHA} values or larger 
\texttt{GAP} values can ameliorate this issue, but the root cause is 
still under investigation. 
\\ \hline

\textbf{800-SON}: Use of global parameters in expressions on .MEASURE lines
will yield incorrect results & The use of global parameters in expressions
on .MEASURE lines is not allowed, as documented in the \Xyce{} Reference Guide.
However, instead of producing a parsing error the measure statement will be
evaluated with the specified qualifier value (e.g., \texttt{FROM}) being left
at its default value.

\textbf{\textit{Workaround}}: None, other than not doing this.
\\ \hline

\textbf{970-SON}: Some devices do not work in frequency-domain analysis &
Devices that may be expected to work in AC or HB analysis do not at
this time.  For AC this includes, but is not limited to, the lossy
transmission line (LTRA) and lossless transmission line (TRA).  For
HB, the transmission lines do work but the nonlinear dependent sources
(B source and nonlinear E, F, G, or H source) do not.

\textbf{\textit{Workaround}}: The LTRA and TRA models will need to be replaced
with lumped transmission line models (YTRANSLINE) for AC analysis.
There is not yet a workaround for the B source in harmonic balance.
\\ \hline

\textbf{967-SON}: Zoltan segmentation fault with OpenMPI 2.1.x and 3.0.0 on
some systems &

It has been observed that when \Xyce{} and Trilinos are built with
OpenMPI 2.1.x or 3.0.0 on certain unsupported operating systems, a
small number of test cases in the regression suite crash with a
segmentation fault inside the Zoltan library.

The \Xyce{} team has determined that this is not a bug in
either \Xyce{} or Zoltan, but is instead due to some pre-packaged OpenMPI
binaries on some operating systems having been built with an
inappropriate option.  This option, ``--enable-heterogeneous'' is
explicitly documented in OpenMPI documentation as broken and unusable
since 2013, but some package managers have OpenMPI binaries built with
this option explicitly enabled.  Turning on this option causes the
resulting OpenMPI build to perform certain communication operations in
a way that does not adhere to the MPI standard.  There is nothing that
can be done in \Xyce{} or Zoltan to fix this issue --- it is entirely
a bug in the OpenMPI library as built on that system.

A new test case has been added to the \Xyce{} test suite in order to
detect this problem.  The test is ``MPI\_Test/bug\_967'', and it will
be run whenever the test suite is invoked with the ``+parallel'' tag
as described in the documentation for the test suite
at \url{https://xyce.sandia.gov/documentation/RunningTheTests.html}.
If this test fails, your system has a broken OpenMPI build that cannot
be used with \Xyce{}.

At the time of this writing, this issue is present in Ubuntu Linux
versions 17.10 and later, and there is an open bug report for it
at \url{https://bugs.launchpad.net/ubuntu/+source/openmpi/+bug/1731938}.

The issue may be present in other distros of Linux that are derived
from Debian (as is Ubuntu), but we cannot confirm this.

\textbf{\textit{Workaround}}:
The only workaround for this problem is to build OpenMPI from source
yourself, and not to include ``--enable-heterogeneous'' in its
configure options.

You should also post a bug report in your operating system's issue
tracker requesting that they rebuild their OpenMPI binaries without
the ``--enable-heterogeneous'' option.  If you are using Ubuntu, you
should register with that issue tracking system and add yourself to
the list of people it affects in the existing bug report (doing so
increases the ``heat'' of the bug, which may increase the likelihood
of it being fixed).

\\ \hline

\textbf{964-SON}: Compatibility of .PRINT TRANADJOINT with .STEP & The use of
\texttt{.PRINT TRANADJOINT} is not compatible with \texttt{.STEP}.  The
resultant \Xyce{} output will not be correct.

\textbf{\textit{Workaround}}: There is none.
\\ \hline

\textbf{932-SON}: Analysis lines do not support expressions for their
operating parameters & The \Xyce{} parser and analysis handlers do not
yet support the use of expressions on netlist analysis lines such
as \texttt{.TRAN}.  The parameters of these analysis lines (such as
stop time for \texttt{.TRAN} or fundamental frequency
for \texttt{.HB}) may only be expressed as literal numbers.

\textbf{\textit{Workaround}}: There is no workaround internal to \Xyce{}.
Use of an external netlist preprocessor would be required. \\ \hline

\textbf{883-SON} .PREPROCESS REPLACEGROUND does not work on nodes referenced
in expressions & The \texttt{.PREPROCESS REPLACEGROUND} feature does
not replace ground synonyms if they appear in B source expressions.

\textbf{\textit{Workaround}}: Do not use ground synonyms (\texttt{GND},
\texttt{GROUND}, etc.) in expressions.  Use a literal ``0'' when
referring to the ground node in expressions.\\ \hline

\textbf{812-SON}: Undocumented limitations on, and bugs with, parameter and
global parameter names & Based on external customer input and
pre-release testing, there are some bugs and undocumented limitations
on parameter and global parameter names in
\Xyce{}. Parameters and global parameters should start with a letter, rather
than with a number or ``special'' character like \#.  In addition, the
use of a single character $V$ as a global parameter name can result in
either netlist parsing failures or incorrect results
from \texttt{.PRINT} lines.  \\ \hline

\textbf{807-SON}: BSIM4 convergence problems with non-zero rgatemod value &
There have been reports of convergence problems (e.g., the \Xyce{}
simulation fails part way through and says that the ``time step is too
small'') when the \texttt{rgatemod} parameter is non-zero. \\ \hline

\textbf{794-SON}: Bug in TABLE Form of \Xyce{} Controlled Sources & In some case, a \Xyce{}
netlist that contains a controlled source that uses the TABLE form will get
the correct answer at first.  However, it may then "stall" (e.g, keep
taking really small time-steps) and never complete the simulation run.

\textbf{\textit{Workaround}}: In some cases, the TABLE specification for the controlled
source can be replaced with a Piecewise Linear (PWL) source that uses
nested IF statements. \\ \hline

\textbf{783-SON}: Use of ddt in a B-Source definition may produce incorrect
results & The \texttt{DDT()} function from the \Xyce{} expression
package, which implements a time derivative, may not function
correctly in a B-Source definition.

\textbf{\textit{Workaround}}: None. \\ \hline

\textbf{727-SON}: \Xyce{} parallel builds hang randomly on OS X & During
Sandia's internal nightly testing of the OSX parallel builds, we see
that \Xyce{} ``hangs on exit'' with an estimated frequency of less
than 1-in-5000 simulation runs.  We have not seen this issue with
parallel builds for either RHEL6 or BSD.  The hang is on exit, whether
on a successful exit or on an error exit.  The hang occurs after all
of the \Xyce{} output has occurred though.  So, the user will get
their sim results, but might have trouble if the individual \Xyce{}
runs are part of a larger script.

\textbf{\textit{Workaround}}: None. \\ \hline

\textbf{661-SON} Lead currents and power accessors (I(), P() and W()) do
not work properly in .RESULT Statements & There are two issues.
First, \texttt{.RESULT} statements will fail netlist parsing if the
requested lead current is omitted from the \texttt{.PRINT TRAN}
line.  As an example, this statement (\texttt{.RESULT I(R1)}) requires
either \texttt{I(R1)},
\texttt{P(R1)} or \texttt{W(R1)} to be on the \texttt{.PRINT TRAN} line.
Second, the output value, in the \texttt{.res} file, for the lead
current or power calculation will always be zero.
\\ \hline

\textbf{583-SON}: Switch with RON=0 leads to convergence failure. &
The switch device does not prevent a user from
specifying \texttt{RON=0} in its model, but then takes the inverse of
this value to get the ``on'' conductance.  The resulting invalid
division will either lead to a division by zero error on platforms
that throw such errors, or produce a conductance with ``Not A Number''
or ``Infinity'' as value.  This will lead to a convergence failure.

\textbf{\textit{Workaround}}: Do not specify an identically zero resistance
for the switch's ``on'' value.  A small value of resistance such as
1e-15 or smaller will generally work well as a substitute. \\ \hline


\textbf{469-SON}: Belos memory consumption on FreeBSD and excessive CPU on other
platforms & Memory or thread bloat can result when using multithreaded
dense linear algebra libraries, which are employed by Belos.  If this
situation is observed, either build
\Xyce{} with a serial dense linear algebra library or use environment variables
to control the number of spawned threads in a multithreaded library.
\\ \hline


\textbf{468-SON}: It should be legal to have two model cards with the same model
name, but different model types. & SPICE3F5 and ngspice only require
that model cards of the same type have unique model names. They accept
model cards of different types with the same name.  \Xyce{} requires
that all model card names be unique.
\\ \hline


\textbf{250-SON}: NODESET in \Xyce{} is not equivalent to NODESET in SPICE & As
currently implemented, \texttt{.NODESET} applies the initial
conditions given throughout a full nonlinear solve for the operating
point, then uses the result as an initial guess for a second nonlinear
solve with no constraints.  This is not the same as SPICE, which
merely applies the given initial conditions to a single nonlinear
solve for the first two iterations, then lets the problem converge
with no further constraints.  This can lead to
a \Xyce{} \texttt{.NODESET} failing where the same netlist in SPICE
might not, if the initial conditions are such that a full nonlinear
solve cannot converge with those constraints in place.  There is no
workaround.
\\ \hline

\textbf{247-SON}: Expressions don't work on .options lines & Expressions
enclosed in braces (\{ \}) are handled specially throughout \Xyce{},
and may only be used in certain contexts such as in device model or
instance parameters or on \texttt{.PRINT} lines.
\\ \hline


\textbf{49-SON} \Xyce{} BSIM models recognize the model TNOM, but not the
instance TNOM & Some simulators allow the model parameter TNOM of BSIM
devices to be specified on the instance line, overriding the model
parameter TNOM.  \Xyce{} does not support this.
\\ \hline


\textbf{27-SON}: Fix handling of .options parameters & When specifying .options
for a particular package, what gets applied as the non-specified
default options might change.  \\ \hline

\textbf{2119-SRN}: Voltages from interface nodes for subcircuits do not 
work in expressions used in device instance parameters & This bug can be
illustrated with this netlist fragment:
\begin{alltt}
X1 1 2 MySub
.SUBCKT MYSUB a c
R1   a b 0.5
R2   b c 0.5
.ENDS
B1 3 0 V=\{V(X1:a)\}
\end{alltt}
This fragment will produce the netlist parsing error \texttt{Directory 
node not found: X1:A}.  The workaround is to use \texttt{V=\{V(1)\}} 
in the B-source expression instead.  This bug also affects the 
solution-dependent capacitor.
\\ \hline

\textbf{1923-SRN}: LC lines run out of memory, even if equivalent (larger) RLC
lines do not. & In some cases, circuits that run fine using an RLC
approximation for a transmission line, exit with an out-of-memory
error if the (supposedly smaller) LC approximation is used.
\\ \hline

\textbf{1595-SRN}: \Xyce{} won't allow access to inductors within subcircuits
for mutual inductors external to subcircuits & It is not possible to
have a mutual inductor outside of a subcircuit couple to inductors in
a subcircuit.

\textbf{\textit{Workaround}}: Put all inductors and mutual inductance lines
that couple to them together at the same level of circuit hierarchy.
\\ \hline

\end{longtable}
}


\newpage
\section{Supported Platforms}
\subsection*{Certified Support}
The following platforms have been subject to certification testing for the
\Xyce{} version 7.5 release.
\begin{XyceItemize}
  \item Red Hat Enterprise Linux${}^{\mbox{\textregistered}}$ 7, x86-64 (serial and parallel)
  \item Microsoft Windows 10${}^{\mbox{\textregistered}}$, x86-64 (serial)
  \item Apple${}^{\mbox{\textregistered}}$ macOS 10.14 and 10.15, x86-64 (serial and parallel)
\end{XyceItemize}


\subsection*{Build Support}
Though not certified platforms, \Xyce{} has been known to run on the following
systems.
\begin{XyceItemize}
  \item FreeBSD 12.X on Intel x86-64 and AMD64 architectures (serial
    and parallel)
  \item Distributions of Linux other than Red Hat Enterprise Linux 6
  \item Microsoft Windows under Cygwin and MinGW.
\end{XyceItemize}


\section{\Xyce{} Release \XyceVersionVar{} Documentation}
The following \Xyce{} documentation is available on the \Xyce{} website in pdf
form.
\begin{XyceItemize}
  \item \Xyce{} Version \XyceVersionVar{} Release Notes (this document)
  \item \Xyce{} Users' Guide, Version \XyceVersionVar{}
  \item \Xyce{} Reference Guide, Version \XyceVersionVar{}
  \item \Xyce{} Mathematical Formulation
  \item Power Grid Modeling with \Xyce{}
  \item Application Note: Coupled Simulation with the \Xyce{} General
    External Interface
  \item Application Note: Mixed Signal Simulation with \Xyce{} 7.2
\end{XyceItemize}
Also included at the \Xyce{} website as web pages are the following.
\begin{XyceItemize}
  \item Frequently Asked Questions
  \item Building Guide (instructions for building \Xyce{} from the source code)
  \item Running the \Xyce{} Regression Test Suite
  \item \Xyce{}/ADMS Users' Guide
  \item Tutorial:  Adding a new compact model to \Xyce{}
\end{XyceItemize}


\section{External User Resources}
\begin{itemize}
  \item Website: {\color{XyceDeepRed}\url{http://xyce.sandia.gov}}
  \item Google Groups discussion forum:
    {\color{XyceDeepRed}\url{https://groups.google.com/forum/#!forum/xyce-users}}
  \item Email support:
    {\color{XyceDeepRed}\href{mailto:xyce@sandia.gov}{xyce@sandia.gov}}
  \item Address:
    \begin{quote}
            Electrical Models and Simulation Dept.\\
            Sandia National Laboratories\\
            P.O. Box 5800, M.S. 1177\\
            Albuquerque, NM 87185-1177 \\
    \end{quote}
\end{itemize}

\vspace*{\fill}
\noindent
Sandia National Laboratories is a multimission laboratory managed and
operated by National Technology and Engineering Solutions of Sandia,
LLC, a wholly owned subsidiary of Honeywell International, Inc., for
the U.S. Department of Energy's National Nuclear Security
Administration under contract DE-NA0003525.

SAND2022-5611 O
\end{document}

