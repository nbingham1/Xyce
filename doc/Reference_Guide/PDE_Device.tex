% Sandia National Laboratories is a multimission laboratory managed and
% operated by National Technology & Engineering Solutions of Sandia, LLC, a
% wholly owned subsidiary of Honeywell International Inc., for the U.S.
% Department of Energy’s National Nuclear Security Administration under
% contract DE-NA0003525.

% Copyright 2002-2024 National Technology & Engineering Solutions of Sandia,
% LLC (NTESS).


Semiconductor device simulation, which is based on a coupled set of partial
differential equations (PDE's) is supported in \Xyce{}.  Such devices can be
invoked from the circuit netlist in a manner similar to traditional SPICE-style
analog devices.  One dimensional and two dimensional devices are supported,
with the dimensionality determined by the device model level.

\begin{Device}

\item[1D Device Form]
\begin{alltt}
YPDE <name> <node> [node] [model name]
+ [device parameters]
\end{alltt}

\vbox{\hrulefill}
\item[2D Device Form]
\begin{alltt}
YPDE <name> <node> <node> [node][node] [model name]|
+ [device parameters]
\end{alltt}

\model
\begin{alltt}
.MODEL <model name> ZOD [model parameters]
\end{alltt}

\comments

All of the PDE parameters are specified on the instance level.  The model
statement is used only for specifying if the device is 1D or 2D, via the level
parameter.  Both the 1D and the 2D devices can construct evenly-spaced meshes
internally, or an unstructured mesh can be read in from an external mesh file.

The eletrode, doping and material parameters are specified using a special
format that is described in the tables that are referenced in the instance
parameter tables.
\end{Device}

\paragraph{TCAD Device Parameters}
Most TCAD device parameters are specified on the instance level.
\noindent
The only TCAD device model parameter is the level, which specifies whether the
model is one or two dimensions.

\label{PDE_Model_Params}
\index{PDE Devices!model parameters}
\begin{DeviceParamTable}{PDE Device Model Parameters}
LEVEL & Determines if the device is 1D or 2D  1=1D, 2=2D & -- & 1 \\ \hline
\end{DeviceParamTable}

\index{PDE Devices!1D instance parameters}
\input{Pde_1_Device_Instance_Params}
\clearpage

\index{PDE Devices!1D instance parameters}
\input{Pde_2_Device_Instance_Params}
\clearpage

\paragraph{Layer Parameters}
\input{LAYER_Composite_Params}

\paragraph{Electrode Parameters 1D}
\input{NODE_Composite_Params}

\paragraph{Electrode Parameters 2D}
\input{NODE_2D_Composite_Params}

\paragraph{Doping or Region Parameters}
The \texttt{DOPINGPROFILES} and \texttt{REGION} parameters are synonyms,
therefore their tables of values are identical.  The use of both parameters in
the same device instance could lead to unpredictable behavior.
\input{DOPINGPROFILES_Composite_Params}
\input{REGION_Composite_Params}

\paragraph{Flat Parameters}
% Sandia National Laboratories is a multimission laboratory managed and
% operated by National Technology & Engineering Solutions of Sandia, LLC, a
% wholly owned subsidiary of Honeywell International Inc., for the U.S.
% Department of Energy’s National Nuclear Security Administration under
% contract DE-NA0003525.

% Copyright 2002-2024 National Technology & Engineering Solutions of Sandia,
% LLC (NTESS).

%%
%% Table describing the flatx, flaty parameters.
%%

%% ERK: Note:  I made each row of the table actually be two rows, because the 
%% graphical cross section picture, which is coming from a jpg file, was 
%% tending to overwrite the lines of the table.  In particular, it was 
%% overwriting the line immediately above it.  It did this for every 
%% (length, width) value that I tried.
%%
\small

\begin{longtable}[htbp]{|
>{\setlength{\hsize}{.2\hsize}}Y|
>{\setlength{\hsize}{.6\hsize}}Y|
>{\setlength{\hsize}{.2\hsize}}Y|} 

\caption[Description of the flatx, flaty doping parameters] {Description of the flatx, flaty doping parameters}
\label{flatxy_table}\\ \hline

\rowcolor{XyceDarkBlue}\color{white}\textbf{flatx or flaty value} 
& \color{white}\bf Description
& \color{white}\bf 1D Cross Section\endhead
      &  &  \\
   0  & Gaussian on both sides of the peak (\texttt{xloc}) location. &   
           {\includegraphics[width=0.300in,height= 0.300in]{flatxy1}}
\\ \hline
      &  &  \\
  +1  & Gaussian if \texttt{x>xloc}, flat (constant at the peak value) if \texttt{x<xloc}. &    
           {\includegraphics[width=0.300in,height= 0.300in]{flatxy2}}
\\ \hline
      &  &  \\
  -1  & Gaussian if \texttt{x<xloc}, flat (constant at the peak value) if \texttt{x>xloc}. &    
           {\includegraphics[width=0.300in,height= 0.300in]{flatxy3}}
\\ \hline

\end{longtable}



