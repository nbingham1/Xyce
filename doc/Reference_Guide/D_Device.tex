% Sandia National Laboratories is a multimission laboratory managed and
% operated by National Technology & Engineering Solutions of Sandia, LLC, a
% wholly owned subsidiary of Honeywell International Inc., for the U.S.
% Department of Energy’s National Nuclear Security Administration under
% contract DE-NA0003525.

% Copyright 2002-2024 National Technology & Engineering Solutions of Sandia,
% LLC (NTESS).


\begin{Device}\label{D_DEVICE}

\symbol
{\includegraphics{diodeSymbol}}

\device
D<name> <(+) node> <(-) node> <model name> [area value]

\model
.MODEL <model name> D [model parameters]

\examples
\begin{alltt}
  DCLAMP 1 0 DMOD
  D2 15 17 SWITCH 1.5
\end{alltt}

\parameters
\begin{Parameters}

\param{\vbox{\hbox{(+) node\hfil}\hbox{(-) node}}}

The anode and the cathode.

\param{area value} 

Scales IS, ISR, IKF, RS, CJO, and IBV, and has a default value of 1.
IBV and BV are both specified as positive values.

\param {PJ value}

Used in computing the junction sidewall effects, and has a default
value of zero (no sidewall effects).

\end{Parameters}

\comments

The diode is modeled as an ohmic resistance (\texttt{RS/area}) in series
with an intrinsic diode.  Positive current is current flowing from the
anode through the diode to the cathode. 

The power through the diode is calculated 
with $I \cdot \Delta V$ where the voltage drop is calculated as $(V_+ - V_-)$ 
and positive current flows from $V_+$ to $V_-$.  This formula may differ from
other simulators, such as HSPICE.

\end{Device}

\paragraph{Diode Operating Temperature}
\index{diode!operating temperature} Model parameters can be assigned unique
measurement temperatures using the \textrmb{TNOM} model parameter.

\paragraph{Diode level selection}

Several distinct implementations of the diode are available.  These are
selected by using the \verb|LEVEL| model parameter.  The default
implementation is based on SPICE 3F5, and may be explicitly specified
using \verb|LEVEL=1| in the model parameters, but is also selected if no
\verb|LEVEL| parameter is specified.  The PSpice implementation
~\cite{PSpiceUG:1998} is obtained by specifying \verb|LEVEL=2|.
The \Xyce{} \verb|LEVEL=200| diode is the JUNCAP200 model.
The \Xyce{} \verb|LEVEL=2002| diode is the DIODE\_CMC model version 2.0.0.


\pagebreak

\paragraph{Level 1 and 2 Diode Instance Parameters}
% This table was generated by Xyce:
%   Xyce -doc D 1
%
\index{diode!device instance parameters}
\begin{DeviceParamTableGenerated}{Diode Device Instance Parameters}{D_1_Device_Instance_Params}
AREA & Area scaling value (scales IS, ISR, IKF, RS, CJ0, and IBV) & -- & 1 \\ \hline
DTEMP & Device delta temperature & $^\circ$C & 0 \\ \hline
IC &  & -- & 0 \\ \hline
M & multiplicity factor & -- & 1 \\ \hline
OFF & Initial voltage drop across device set to zero & logical (T/F) & 0 \\ \hline
PJ & Perimeter scaling value & -- & 0 \\ \hline
TEMP & Device temperature & -- & Ambient Temperature \\ \hline
\end{DeviceParamTableGenerated}


\paragraph{Level 1 and 2 Diode Model Parameters}
\input{D_1_Device_Model_Params}

\paragraph{JUNCAP200 (level=200) Parameters}
The JUNCAP200 model has the instance and model parameters in the
tables below.  Complete documentation of JUNCAP200 may be found at
\url{http://www.cea.fr/cea-tech/leti/pspsupport/Documents/juncap200p5_summary.pdf}.


The JUNCAP200 device supports output of the internal variables in
table~\ref{D_200_OutputVars} on the \texttt{.PRINT} line of a netlist.
To access them from a print line, use the syntax
\texttt{N(<instance>:<variable>)} where ``\texttt{<instance>}'' refers to the
name of the specific level 200 D device in your netlist.

\input{D_200_Device_Instance_Params}
\input{D_200_Device_Model_Params}
\input{D_200_OutputVars}

\paragraph{DIODE\_CMC (level=2002) Parameters}
The DIODE\_CMC model has the instance and model parameters in the
tables below.  Complete documentation of DIODE\_CMC may be found at
\url{https://si2.org/standard-models}.


The DIODE\_CMC device supports output of the internal variables in
table~\ref{D_2002_OutputVars} on the \texttt{.PRINT} line of a netlist.
To access them from a print line, use the syntax
\texttt{N(<instance>:<variable>)} where ``\texttt{<instance>}'' refers to the
name of the specific level 2002 D device in your netlist.

\input{D_2002_Device_Instance_Params}
\input{D_2002_Device_Model_Params}
\input{D_2002_OutputVars}

\paragraph{Level 1 Diode Equations}

The equations in this section use the following variables:
\begin{eqnarray*}
V_{di} & = & \mbox{voltage across the intrinsic diode only} \\
V_{th} & = & \mbox{$k \cdot T/q$ (thermal voltage)}         \\
k      & = & \mbox{Boltzmann's constant}                    \\
q      & = & \mbox{electron charge}                         \\
T      & = & \mbox{analysis temperature (Kelvin)}           \\
T_{0}  & = & \mbox{nominal temperature (set using \textrmb{TNOM}
option)} \\
\omega & = & \mbox{Frequency (Hz)}
\end{eqnarray*}
Other variables are listed above in the diode model parameters.

\subparagraph{Level=1}
The level 1 diode is based on the Spice3f5 level 1 model.

\subparagraph{DC Current (Level=1)}

The intrinsic diode current consists of forward and reverse bias regions where
$$
I_D = \left\{ \begin{array}{ll}
\mathbf{IS}\cdot\left[\exp \left(\frac{V_{di}}{\mathbf{N}V_{th}}\right) - 1
\right], & V_{di} > -3.0\cdot\mathbf{N}V_{th} \\
-\mathbf{IS}\cdot\left[1.0 + \left(\frac{3.0\cdot\mathbf{N}V_{th}}{V_{di}\cdot
e}\right)^3\right], & V_{di} < -3.0\cdot\mathbf{N}V_{th}
\end{array}
\right.
$$


When $\bBV$ and an optional parameter $\bIBV$ are explicitly given in the model
statement, an exponential model is used to model reverse breakdown (with a
``knee'' current of $\bIBV$ at a ``knee-on'' voltage of $\bBV$).  The equation
for $I_D$ implemented by \Xyce{} is given by

\[
I_D = -\bIBVeff\cdot\exp \left(-\frac{\bBVeff + V_{di}}{\mathbf{N}V_{th}}
\right), \hspace*{0.5in} V_{di} \leq \bBVeff,
\]
where $\bBVeff$ and $\bIBVeff$ are chosen to satisfy the following
constraints:
\begin{enumerate}
\item Continuity of $I_D$ between reverse bias and reverse breakdown regions
(i.e., continuity of $I_D$ at $V_{di}=-\bBVeff$):
\[
\bIBVeff=\bIS\left(1-\left(\frac{3.0\cdot\bN V_{th}}{e\cdot\bBVeff}
\right)^3\right)
\]
\item ``Knee-on'' voltage/current matching:
\[
\bIBVeff\cdot\exp\left(-\frac{\bBVeff-\bBV}{\bN V_{th}}\right)=\bIBV
\]
\end{enumerate}
Substituting the first expression into the second yields a single constraint
on $\bBVeff$ which cannot be solved for directly.  By performing some basic
algebraic manipulation and rearranging terms, the problem of finding
$\bBVeff$ which satisfies the above two constraints can be cast as finding
the (unique) solution of the equation
\begin{equation}
\bBVeff = f(\bBVeff),
\label{eqn:diode_picardeqn}
\end{equation}
where $f(\cdot)$ is the function that is obtained by solving for the
$\bBVeff$ term which appears in the exponential in terms of $\bBVeff$ and the
other parameters.  \Xyce{} solves Eqn.\ \ref{eqn:diode_picardeqn} by performing
the so-called {\em Picard Iteration} procedure \cite{mattuck:1999}, i.e.
by producing successive estimates of $\bBVeff$ (which we will denote as
$\bBVeff ^ k$) according to
\[
\bBVeff^{k+1}=f(\bBVeff^k)
\]
starting with an initial guess of $\bBVeff^0=\bBV$.  The current iteration
procedure implemented in \Xyce{} can be shown to guarantee at least six
significant digits of accuracy between the numerical estimate of $\bBVeff$ and
the true value.

In addition to the above, \Xyce{} also requires that $\bBVeff$ lie in the
range \mbox{$\bBV \geq \bBVeff \geq 3.0\bN V_{th}$.}  In terms of $\bIBV   $,
this is equivalent to enforcing the following two constraints:
\begin{eqnarray}
\bIS\left(1-\left(\frac{3.0\cdot\bN V_{th}}{e\cdot\bBV}\right)^3\right) & \leq & \bIBV \label{eqn:diode_ibvconstr1}\\
\bIS\left(1-e^{-3}\right)\exp\left(\frac{-3.0\cdot\bN V_{th}+
\bBV}{\bN V_{th}}\right) & \geq & \bIBV \label{eqn:diode_ibvconstr2}
\end{eqnarray}
\Xyce{} first checks the value of $\bIBV$ to ensure that the above two
constraints are satisfied.  If Eqn.\ \ref{eqn:diode_ibvconstr1} is violated,
\Xyce{} sets $\bIBVeff$ to be equal to the left-hand side of Eqn.\
\ref{eqn:diode_ibvconstr1} and, correspondingly, sets $\bBVeff$ to $-3.0\cdot
\bN V_{th}.$  If Eqn.\ \ref{eqn:diode_ibvconstr2} is violated, \Xyce{} sets
$\bIBVeff$ to be equal to the left-hand side of Eqn.\
\ref{eqn:diode_ibvconstr2} and, correspondingly, sets $\bBVeff$ to $\bBV$.

\subparagraph{Capacitance (Level=1)}
The p-n diode capacitance consists of a depletion layer capacitance $C_d$ and a
diffusion capacitance $C_{dif}$.  The first is given by
\[
C_d = \left\{ \begin{array}{ll}
\mathbf{CJ}\cdot\mathbf{AREA} \left(1-\frac{V_{di}}{\mathbf{VJ}} \right)^{-\mathbf{M}}, &
V_{di} \leq \mathbf{FC \cdot VJ} \\
\frac{\mathbf{CJ}\cdot\mathbf{AREA}}{\mathbf{F2}}\left(\mathbf{F3}+\mathbf{M}\frac{V_{di}}{\mathbf{VJ}}\right),
& V_{di} > \mathbf{FC \cdot VJ}
\end{array}
\right.
\]
The diffusion capacitance (sometimes referred to as the transit time
capacitance) is
\[
C_{dif} = \mathbf{TT}G_d = \mathbf{TT}\frac{dI_D}{dV_{di}}
\]
where $G_d$ is the junction conductance.

\subparagraph{Sidewall currents and capacitances}
When the instance parameter \texttt{PJ} (perimeter scaling value) is
specified, the diode currents become the sum of the currents above
(the ``bottom'' of the junction) and those of the periphery (sidewall).

In normal forward and reverse bias regions, the sidewall currents are given by:
$$
I_{D,SW} = \left\{ \begin{array}{ll}
\mathbf{ISatSW}\cdot\left[\exp \left(\frac{V_{di}}{\mathbf{NS}V_{th}}\right) - 1
\right], & V_{di} > -3.0\cdot\mathbf{NS}V_{th} \\
-\mathbf{ISatSW}\cdot\left[1.0 + \left(\frac{3.0\cdot\mathbf{NS}V_{th}}{V_{di}\cdot
e}\right)^3\right], & V_{di} < -3.0\cdot\mathbf{NS}V_{th}
\end{array}
\right.
$$

where $\mathbf{ISatSW}$ is the temperature-adjusted value of JSW multiplied
by the perimeter $\mathbf{PJ}$.

When the breakdown voltage \texttt{BV} has been given and the diode voltage is below $-\mathbf{BV}$, the sidewall current is:
\[
I_{D,sw} = -\mathbf{ISatSW}\cdot\exp \left(-\frac{\mathbf{BVeff} + V_{di}}{\mathbf{NS}V_{th}}
\right), \hspace*{0.5in} V_{di} \leq \mathbf{BVeff},
\]

The sidewall capacitances are computed as:
\[
C_{d,sw} = \left\{ \begin{array}{ll}
\mathbf{CJSW}\cdot\mathbf{PJ} \left(1-\frac{V_{di}}{\mathbf{PHP}} \right)^{-\mathbf{M}}, &
V_{di} \leq \mathbf{FCS \cdot PHP} \\
\frac{\mathbf{CJSW}\cdot\mathbf{PJ}}{\mathbf{F2SW}}\left(\mathbf{F3SW}+\mathbf{MJSW}\frac{V_{di}}{\mathbf{PHP}}\right),
& V_{di} > \mathbf{FCS \cdot PHP}
\end{array}
\right.
\]


\subparagraph{Temperature Effects (Level=1)}
The diode model contains explicit temperature dependencies in the ideal diode
current, the generation/recombination current and the breakdown current.
Further temperature dependencies are present in the diode model via the
saturation current $I_{S}$, the depletion layer junction capacitance $CJ$,
the junction potential $V_J$.
%insert equations here (3)
\begin{eqnarray*}
V_t(T) & = & \frac{kT}{q} \\
V_{tnom}(T) & = & \frac{k\mathbf{TNOM}}{q} \\
E_g(T) & = & E_{g0} - \frac{\alpha T^2}{\beta + T} \\
E_{gNOM}(T) & = & E_{g0} - \frac{\alpha\mathbf{TNOM}^2}{\mathbf{TNOM}+\beta} \\
arg1(T) & = & -\frac{E_g(T)}{2kT} + \frac{E_{g300}}{2kT_0} \\
arg2(T) & = & -\frac{E_{gNOM}(T)}{2k\mathbf{TNOM}} + \frac{E_{g300}}{2kT_0} \\
pbfact1(T) & = & -2.0\cdot V_t(T) \left(1.5\cdot\ln \left(\frac{T}{T_0}\right) + q\cdot arg1(T)\right) \\
pbfact2(T) & = & -2.0\cdot V_{tnom}(T) \left(1.5\cdot\ln \left(\frac{\mathbf{TNOM}}{T_0}\right) + q\cdot arg2(T)\right) \\
pbo(T) & = & \left(\mathbf{VJ}-pbfact2(T)\right)\frac{T_0}{\mathbf{TNOM}} \\
V_J(T) & = & pbfact1(T) + \frac{T}{T_0}pbo(T) \\
gma_{old}(T) & = & \frac{\mathbf{VJ}-pbo(T)}{pbo(T)} \\
gma_{new}(T) & = & \frac{V_J(T)-pbo(T)}{pbo(T)} \\
CJ(T) & = & \mathbf{CJ0}\frac{1.0+\mathbf{M}\left(4.0\times 10^{-4}\left(T-T_0\right)-gma_{new}(T)\right)}{1.0 + \mathbf{M}\left(4.0\times 10^{-4}\left(\mathbf{TNOM}-T_0\right)-gma_{old}(T)\right)} \\
I_S(T) & = & \mathbf{IS} \cdot\exp \left(\left(\frac{T}{\mathbf{TNOM}}-1.0\right) \cdot \frac{\mathbf{EG}}{\mathbf{N}V_t(T)} + \frac{\mathbf{XTI}}{\mathbf{N}} \cdot \ln \left(\frac{T}{\mathbf{TNOM}}\right)\right) \\
\end{eqnarray*}
where, for silicon, $\alpha = 7.02\times 10^{-4}\;eV/K$, $\beta =
1108\; K$ and $E_{g0} = 1.16\;eV$.

%\subparagraph{Noise}
%There is no noise model in this version of \Xyce{}.
%Noise is calculated with a 1.0 Hz bandwidth using the following spectral power
%densities (per unit bandwidth).
%
%Thermal Noise due to Parasitic Resistance: $I_n^2 = \frac{4kT}
%{\mathbf{RS}/\mbox{area}}$ \\
%Intrinsic Diode Shot and Flicker Noise: $I_n^2 = 2qI_D + \mathbf{KF} \cdot
%\frac{I_D^{\mathbf{AF}}}{\omega}$
%
For a more thorough description of p-n junction physics, see [9].  For a
thorough description of the U.C. Berkeley SPICE models see Reference [11].
