% Sandia National Laboratories is a multimission laboratory managed and
% operated by National Technology & Engineering Solutions of Sandia, LLC, a
% wholly owned subsidiary of Honeywell International Inc., for the U.S.
% Department of Energy’s National Nuclear Security Administration under
% contract DE-NA0003525.

% Copyright 2002-2024 National Technology & Engineering Solutions of Sandia,
% LLC (NTESS).


The \texttt{.LIB} command is similar to \texttt{.INCLUDE}, in that it
brings in an external file.  However, it is
designed to only bring in specific parts of a library file, as designated
by an entry name.
Note that the \Xyce{} version of \texttt{.LIB} has been designed to be compatible
with HSPICE~\cite{Hspice}, not PSpice~\cite{Pspice}.

There are two forms of the \texttt{.LIB} statement, the call and the
definition.  The call statement reads in a specified subset of a
library file, and the definition statement defines the subsets.

\subsubsection{.LIB call statement}
\begin{Command}
\format
.LIB <file name> <entry name>

\examples
\begin{alltt}
.LIB models.lib nom
.LIB 'models.lib'  low
.LIB "models.lib"  low
.LIB "path/models.lib"  high
\end{alltt}

\arguments

\begin{Arguments}

\argument{file name}
Name of file containing netlist data.  Single or double quotes
(\texttt{"} or \texttt{'}) may be used around the file name.

\argument{entry name}
Entry name, which determines the section of the file to be included.
These sections are defined in the included file using the definition
form of the \texttt{.LIB} statement.

\end{Arguments}
\end{Command}

The library file name can be surrounded by quotes (single or double),
as in "path/filename" but this is not necessary.  The directory for
the library file is assumed to be the execution directory unless a
full or relative path is given as a part of the file name.  The
section name denotes the section or sections of the library file to
include.

If {\texttt <file name>} uses an absolute path then that path is used.
Otherwise, the search-path order for {\texttt <file name>} is:
\begin{XyceItemize}
  \item Relative to the directory that contains {\texttt <file name>}.
  \item Relative to the directory that contains the file with the top-level
netlist.
  \item Relative to the \Xyce{} execution directory.
\end{XyceItemize}

\subsubsection{.LIB definition statement}
The format given above is when the \texttt{.LIB} command is used to reference
a library file; however, it is also used as part of the syntax in a library
file. 

\begin{Command}
\format
\begin{alltt}
.LIB <entry name>
<netlist lines>*
.endl <entry name>
\end{alltt}

\examples
\begin{alltt}

* Library file res.lib
.lib low
.param rval=2
r3  2  0  9
.endl low

.lib nom
.param rval=3
r3  2  0  8
.endl nom
\end{alltt}

\arguments
\begin{Arguments}
\argument{entry name}
The name to be used to identify this library component.  When used on a \texttt{.LIB} call line, these segments of the library file will be included in the calling file.
\end{Arguments}
\end{Command}

Note that for each entry name, there is a matched \texttt{.lib}
and \texttt{.endl}.  Any valid netlist commands can be placed inside
the \texttt{.lib} and \texttt{.endl} statements.  The following is an
example calling netlist, which refers to the library in the examples above:

\begin{centering}
\shadowbox{
\begin{minipage}{0.8\textwidth}
\begin{vquote}
\color{blue}* Netlist file res.cir\color{black}
V1  1   0   1
R   1   2   \{rval\}
.lib res.lib nom
.tran 1 ps 1ns
.end
\end{vquote}
\end{minipage}
}
\end{centering}

In this example, only the netlist commands that are inside of the
``nom'' library will be parsed, while the commands inside of the
``low'' library will be discarded.  As a result, the value for
resistor r3 is 8, and the value for rval is 3.
