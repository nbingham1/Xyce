% Sandia National Laboratories is a multimission laboratory managed and
% operated by National Technology & Engineering Solutions of Sandia, LLC, a
% wholly owned subsidiary of Honeywell International Inc., for the U.S.
% Department of Energy’s National Nuclear Security Administration under
% contract DE-NA0003525.

% Copyright 2002-2024 National Technology & Engineering Solutions of Sandia,
% LLC (NTESS).


\index{\texttt{.SAMPLING}}
\index{analysis!sampling} \index{sampling analysis}
Calculates a full analysis (\verb|.DC|, \verb|.TRAN|, \verb|.AC|, etc.) over a distribution of
parameter values.  Sampling operates similarly to \verb|.STEP|, except that the parameter
values are generated from random distributions rather than sweeps.  
If used in conjunction 
with projection-based PCE methods, then the sample points are not based on random samples.  
Instead they are based on the quadrature points.

\index{analysis!SAMPLING} 
\index{SAMPLING analysis}
\index{analysis!MC} 
\index{MC analysis}
\index{analysis!PCE}
\index{PCE analysis}
\index{analysis!Monte Carlo} 
\index{Monte Carlo analysis}
\index{analysis!LHS} 
\index{LHS analysis}
\index{Latin Hypercube Sampling analysis}

\begin{Command}
\format
.SAMPLING  \\
+ param=<parameter name>,[parameter name]*  \\
+ type=<parameter type>,[parameter type]*  \\
+ means=<mean>,[mean]*  \\
+ std\_deviations=<standard deviation>,[standard deviation]* 

\examples
\begin{alltt}
.SAMPLING
+ param=R1
+ type=normal
+ means=3K
+ std\_deviations=1K

.SAMPLING
+ param=R1,R2
+ type=uniform,uniform
+ lower\_bounds=1K,2K
+ upper\_bounds=5K,6K

.SAMPLING
+ useExpr=true

.options SAMPLES numsamples=10000

.options SAMPLES numsamples=25000
+ OUTPUTS=\{R1:R\},\{V(1)\}
+ SAMPLE\_TYPE=MC

.options SAMPLES numsamples=1000
+ MEASURES=maxSine
+ SAMPLE\_TYPE=LHS

.options samples numsamples=30
+ covmatrix=1e6,1.0e-3,1.0e-3,4e-14
+ OUTPUTS=\{V(1)\},\{R1:R\},\{C1:C\}
\end{alltt}

\arguments

\begin{Arguments}

\argument{param}
Names of the parameters to be sampled.  This may be any of the parameters that are valid 
for \verb|.STEP|, including device instance, device model, or global parameters.  
  If more than one parameter, then specify as a comma-separated list.

\argument{type}
Distribution type for each parameter.  This may be uniform or normal.  
  If more than one parameter, then specify as a comma-separated list.

\argument{means}
If using normal distributions, the mean for each parameter must be specified.
  If more than one parameter, then specify as a comma-separated list.

\argument{std\_deviations}
If using normal distributions, the standard deviation for each parameter must be specified.
  If more than one parameter, then specify as a comma-separated list.

\argument{lower\_bounds}
If using uniform distributions, the lower bound must be specified.  This is optional for normal distributions.  
  If used with normal distributions, may alter the mean and standard deviation.
  If more than one parameter, then specify as a comma-separated list.

\argument{upper\_bounds}
If using uniform distributions, the upper bound must be specified.  This is optional for normal distributions.
  If used with normal distributions, may alter the mean and standard deviation.
  If more than one parameter, then specify as a comma-separated list.

\argument{useExpr}
If this argument is set to true, then the sampling algorithm will set up random 
  inputs from expression operators such as \verb|AGAUSS| and \verb|AUNIF|.  In 
  this case it will also ignore the list of parameters on the \verb|.SAMPLING| command line.
  For a complete description of expression-based random operators, see the expression
  documentation in section~\ref{ExpressionDocumentation}.

\end{Arguments}

\comments

In addition to the \verb|.SAMPLING| command, this analysis requires a 
  \verb|.options SAMPLES| command as well.  The \verb|.SAMPLING| command specifies 
  parameters and their attributes, either using the \verb|useExpr| option, or 
  with comma-separated lists.  The \verb|.options SAMPLES| command specifies 
  analysis options, including the number of samples, the type of sampling (LHS or MC)
  and the outputs and/or measures for which to compute statistics.
This line also allows one to specify a non-intrusive Polynomial Chaos 
Expansion (PCE) method (either regression or projection PCE).  
To see the details of the \verb|.options SAMPLES| command , see table~\ref{SamplesPKG}.

On the \verb|.SAMPLING| command line, if not using \verb|useExpr|, 
  parameters and their attributes must be specified 
  using comma-separated lists.  The comma-separated lists must all be the same length.

\end{Command}

