% Sandia National Laboratories is a multimission laboratory managed and
% operated by National Technology & Engineering Solutions of Sandia, LLC, a
% wholly owned subsidiary of Honeywell International Inc., for the U.S.
% Department of Energy’s National Nuclear Security Administration under
% contract DE-NA0003525.

% Copyright 2002-2024 National Technology & Engineering Solutions of Sandia,
% LLC (NTESS).

\label{PCE_section}
\index{\texttt{.PCE}}
\index{analysis!pce} 
Calculates a fully intrusive Polynommial Chaos Expansion (PCE) analysis 
(for \verb|.DC| or \verb|.TRAN| only) 
to propagate uncertainty from a set of uncertain inputs to uncertain outputs.
This involves evaluating the circuit at a set of parameter values corresponding to
quadrature points used by the PCE algorithm.
The loop over parameter values happens at the inner-most part of the calculation, 
so all samples are propagated simultaneously.

This fully-intrusive form of PCE is an experimental analysis method.  Non-intrusive 
methods of PCE are also available in \Xyce{} and will usually be a better choice.  
The non-intrusive methods are used in combination with \texttt{.SAMPLING} 
and/or \texttt{.EMBEDDEDSAMPLING}.  Of those other methods, the behavior of \texttt{.PCE} 
most closely resembles that of \texttt{.EMBEDDEDSAMPLING} with \texttt{projection\_pce=true}.
\index{analysis!PCE}
\index{PCE analysis}

\index{analysis!PCE}
\index{PCE analysis}

\begin{Command}
\format
.PCE  \\
+ param=<parameter name>,[parameter name]*  \\
+ type=<parameter type>,[parameter type]*  \\
+ means=<mean>,[mean]*  \\
+ std\_deviations=<standard deviation>,[standard deviation]* \\
+ lower\_bounds=<lower bound>,[lower bound]*  \\
+ upper\_bounds=<upper bound>,[upper bound]* \\
+ alpha=<alpha>,[alpha]*  \\
+ beta=<beta>,[beta]*

\examples
\begin{alltt}
.PCE
+ param=R1
+ type=normal
+ means=3K
+ std\_deviations=1K

.PCE
+ param=R1,R2
+ type=uniform,uniform
+ lower\_bounds=1K,2K
+ upper\_bounds=5K,6K

.PCE useExpr=true

.options PCES 
+ OUTPUTS=\{R1:R\},\{V(1)\}
\end{alltt}

\arguments

\begin{Arguments}

\argument{param}
Names of the parameters to be sampled.  This may be any of the parameters
that are valid for \verb|.STEP|, including device instance, device model,
or global parameters.  If more than one parameter, then specify as a
comma-separated list.

\argument{type}
Distribution type for each parameter.  This may be uniform, normal or gamma.
If more than one parameter, then specify as a comma-separated list.

\argument{means}
If using normal distributions, the mean for each parameter must be specified.
If more than one parameter, then specify as a comma-separated list.

\argument{std\_deviations}
If using normal distributions, the standard deviation for each parameter
must be specified.  If more than one parameter, then specify as a
comma-separated list.

\argument{lower\_bounds}
If using uniform distributions, the lower bound must be specified.
This is optional for normal distributions.  If used with normal
distributions, may alter the mean and standard deviation.
If more than one parameter, then specify as a comma-separated list.

\argument{upper\_bounds}
If using uniform distributions, the upper bound must be specified.
This is optional for normal distributions.  If used with normal
distributions, may alter the mean and standard deviation.
If more than one parameter, then specify as a comma-separated list.

\argument{alpha}
If using gamma distributions, the alpha value for each parameter
must be specified.  If more than one parameter, then specify as a
comma-separated list.

\argument{beta}
If using gamma distributions, the beta value for each parameter
must be specified.  If more than one parameter, then specify as a
comma-separated list.

\argument{useExpr}
If this argument is set to true, then the sampling algorithm will set up random 
  inputs from expression operators such as \verb|AGAUSS| and \verb|AUNIF|.  In 
  this case it will also ignore the list of parameters on the \verb|.PCE| command line.
  For a complete description of expression-based random operators, see the expression
  documentation in section~\ref{ExpressionDocumentation}.

\end{Arguments}

\comments

In addition to the \verb|.PCE| command, this analysis
requires a \verb|.options PCES| command as well.  The
\verb|.PCE| command specifies parameters and their
attributes, either using the \verb|useExpr| option, or with 
comma-separated lists.  The \verb|.options PCES| command specifies
the outputs for which to compute statistics.
To see the details of the \verb|.options PCES| command , see table~\ref{PCEPKG}.

On the \verb|.PCE| command line, if not using \verb|useExpr|, 
parameters and their
attributes must be specified using comma-separated lists. The
comma-separated lists must all be the same length.

The \texttt{.PRINT PCE} command provides output based on the contents
of those print-lines, and also the \texttt{OUTPUT}
arguments on the \texttt{.OPTIONS PCES} line. 

If the \texttt{OUTPUT\_SAMPLE\_STATS} argument on a \texttt{.PRINT PCE} line is
set to ``true'' then the statistics for the \texttt{MEAN}, \texttt{MEANPLUS},
\texttt{MEANMINUS}, \texttt{STDDEV} and \texttt{VARIANCE} will be output for each
variable in the \texttt{OUTPUT} argument.  
If the \texttt{OUTPUT\_ALL\_SAMPLES}
argument on a \texttt{.PRINT PCE} line is set to ``true'' then the values
of all quadrature points, for each variable requested
in the \texttt{OUTPUTS} argument, will be output.

\end{Command}

