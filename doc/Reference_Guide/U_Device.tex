% Sandia National Laboratories is a multimission laboratory managed and
% operated by National Technology & Engineering Solutions of Sandia, LLC, a
% wholly owned subsidiary of Honeywell International Inc., for the U.S.
% Department of Energy’s National Nuclear Security Administration under
% contract DE-NA0003525.

% Copyright 2002-2024 National Technology & Engineering Solutions of Sandia,
% LLC (NTESS).

 
%%
%% Behavioral Digital Description Table
%%
\begin{Device}\label{U_DEVICE}

 
\device
\begin{alltt}
U<name> <type>(<num inputs>) [digital power node] 
+ [digital ground node] <input node>* <output node>* 
+ <model name> [device parameters]
\end{alltt}
 
\model
.MODEL <model name> DIG [model parameters]
 
\examples
\begin{alltt}
UMYAND AND(2) DPWR DGND in1 in2 out DMOD IC=TRUE
UTHEINV INV DPWR DGND in out DMOD
.model DMOD DIG (
+ CLO=1e-12  CHI=1e-12
+ S0RLO=5  S0RHI=5  S0TSW=5e-9
+ S0VLO=-1  S0VHI=1.8
+ S1RLO=200  S1RHI=5  S1TSW=5e-9
+ S1VLO=1  S1VHI=3
+ RLOAD=1000
+ CLOAD=1e-12
+ DELAY=20ns )
\end{alltt}

\parameters 
\begin{Parameters}

\param{type} 

Type of digital device.  Supported devices are: INV, BUF, AND, NAND, OR, NOR, XOR,
NXOR, DFF, JKFF, TFF, DLTCH and ADD.  (Note: NOT is an allowed synonym for INV, but will be
deprecated in future \Xyce{} releases.)

The following gates have a fixed number of inputs.  INV and BUF have only one
input and one output node.  XOR and NXOR have two inputs and one output.
ADD has three inputs (in1, in2, carryIn) and two outputs (sumOut
and carryOut).  DFF has four inputs (PREB, CLRB, Clock and Data) and
two outputs ($Q$ and $\bar{Q}$).  TFF has two inputs (T and CLK) and two
outputs ($Q$ and $\bar{Q}$).  The TFF uses ``positive'' (``rising'') edge clocking.
The JKFF has five inputs (PREB, CLRB, Clock, J and K) and two outputs ($Q$ and $\bar{Q}$).
The JKFF uses ``negative'' (``falling'') edge clocking.
DLTCH has four inputs (PREB, CLRB, Enable and Data) and two outputs ($Q$ and $\bar{Q}$).  

The AND, NAND, OR and NOR gates have one output but a variable number of inputs.  
There is no limit on the number of inputs for AND, NAND, OR and NOR gates, but
there must be at least two inputs.

\param{num inputs}

For AND, NAND, OR and NOR gates, with N inputs, the syntax is (N), as shown for the
MYAND example given above, where AND(2) is specified.  The inclusion of (N) is mandatory
for gates with a variable number of inputs, and both the left and right parentheses must
be used to enclose N.  

This parameter is optional, and typically omitted, for gates with a fixed number of
inputs, such as INV, BUF, XOR, NXOR, DFF, JKFF, TFF, DLTCH and ADD.  This is illustrated by the THEINV
example given above, where the device type is INV rather than INV(1).

\param{digital power node}

Dominant node to be connected to the output node(s) to establish high
output state.  This node is connected to the output by a resistor and
capacitor in parallel, whose values are set by the model.  This node must
be specified on the instance line.

\param{digital ground node}

This node serves two purposes, and must be specified on the instance line.
It is the dominant node to be connected to the output node(s) to establish
low output state.  This node is connected to the output by a resistor and 
capacitor in parallel, whose values are set by the model.  This node is also
connected to the input node by a resistor and capacitor in parallel, whose
values are set by the model.  Determination of the input state is based on the
voltage drop between the input node and this node.

\param{input nodes, output nodes}

Input and output nodes that connect to the circuit.

\param{model name}

 Name of the model defined in a .MODEL line.

\param{device parameters}

Parameter listed in Table~\ref{Digital_1_Device_Instance_Params_Udevice} may be
provided as \texttt{<parameter>=<value>} specifications as needed.  For
devices with more than one output, multiple output initial states may be
provided as Boolean values in either a comma separated list (e.g.
IC=TRUE,FALSE for a device with two outputs) or individually 
(e.g. IC1=TRUE IC2=FALSE or IC2=FALSE).  Finally, the IC specification
must use TRUE and FALSE rather than T and F.

\end{Parameters}
\end{Device}

\paragraph{Device Parameters}

%%
%% Digital Device Param Table
%%
% Sandia National Laboratories is a multimission laboratory managed and
% operated by National Technology & Engineering Solutions of Sandia, LLC, a
% wholly owned subsidiary of Honeywell International Inc., for the U.S.
% Department of Energy’s National Nuclear Security Administration under
% contract DE-NA0003525.

% Copyright 2002-2024 National Technology & Engineering Solutions of Sandia,
% LLC (NTESS).

% This table was generated by Xyce:
%   Xyce -doc Digital 1
%
\index{behavioral digital!device instance parameters}
\begin{DeviceParamTableGenerated}{Behavioral Digital Device Instance Parameters}{Digital_1_Device_Instance_Params_Udevice}
IC1 & Vector of initial values for output(s) & logical (T/F) & false \\ \hline
IC2 &  & -- & false \\ \hline
\end{DeviceParamTableGenerated}


\paragraph{Model Parameters}

%%
%% Digital Model Param Table
%%
% Sandia National Laboratories is a multimission laboratory managed and
% operated by National Technology & Engineering Solutions of Sandia, LLC, a
% wholly owned subsidiary of Honeywell International Inc., for the U.S.
% Department of Energy’s National Nuclear Security Administration under
% contract DE-NA0003525.

% Copyright 2002-2024 National Technology & Engineering Solutions of Sandia,
% LLC (NTESS).

% This table was generated by Xyce:
%   Xyce -doc Digital 1
%
\index{behavioral digital!device model parameters}
\begin{DeviceParamTableGenerated}{Behavioral Digital Device Model Parameters}{Digital_1_Device_Model_Params_Udevice}
CHI & Capacitance between output node and high reference & F & 1e-06 \\ \hline
CLO & Capacitance between output node and low reference & F & 1e-06 \\ \hline
CLOAD & Capacitance between input node and input reference & F & 1e-06 \\ \hline
DELAY & Delay time of device & s & 1e-08 \\ \hline
RLOAD & Resistance between input node and input reference & $\mathsf{\Omega}$ & 1000 \\ \hline
S0RHI & Low state resitance between output node and high reference & $\mathsf{\Omega}$ & 100 \\ \hline
S0RLO & Low state resistance between output node and low reference & $\mathsf{\Omega}$ & 100 \\ \hline
S0TSW & Switching time transition to low state & s & 1e-08 \\ \hline
S0VHI & Maximum voltage to switch to low state & V & 1.7 \\ \hline
S0VLO & Minimum voltage to switch to low state & V & -1.5 \\ \hline
S1RHI & High state resistance between output node and high reference & $\mathsf{\Omega}$ & 100 \\ \hline
S1RLO & High state resistance between output node and low reference & $\mathsf{\Omega}$ & 100 \\ \hline
S1TSW & Switching time transition to high state & s & 1e-08 \\ \hline
S1VHI & Maximum voltage to switch to high state & V & 7 \\ \hline
S1VLO & Minimum voltage to switch to high state & V & 0.9 \\ \hline
\end{DeviceParamTableGenerated}


\paragraph{Model Description}

The input interface model consists of the input node connected with a resistor and
capacitor in parallel to the digital ground node.  The values of these are: \textrmb{RLOAD}
and \textrmb{CLOAD}.  

The logical state of any input node is determined by comparing the voltage relative
to the reference to the range for the low and high state.  The range for the low
state is \textrmb{S0VLO} to \textrmb{S0VHI}.  Similarly, the range for the high state
is \textrmb{S1VLO} to \textrmb{S1VHI}.  The state of an input node will remain fixed as
long as its voltage stays within the range for its current state.  That input node will
transition to the other state only when its state goes outside the voltage range of
its current state.

The output interface model is more complex than the input model, but shares the same
basic configuration of a resistor and capacitor in parallel to simulate loading.  For
the output case, there are such parallel RC connections to two nodes, the digital
ground node and the digital power node.  Both of these nodes must be specified on
the instance line.

The capacitance to the high node is specified by \textrmb{CHI}, and the capacitance to the low
node is \textrmb{CLO}.  The resistors in parallel with these capacitors are variable, and have
values that depend on the state.  In the low state (S0), the resistance values are:
\textrmb{S0RLO} and \textrmb{S0RHI}.  In the high state (S1) ,the resistance values are: 
\textrmb{S1RLO} and \textrmb{S1RHI}.  Transition to the high state occurs exponentially over
a time of \textrmb{S1TSW}, and to the low state \textrmb{S0TSW}.

The device's delay is given by the model parameter \textrmb{DELAY}.  Any input changes
that affect the device's outputs are propagated after this delay.

As a note, the model parameters \textrmb{VREF}, \textrmb{VLO} and \textrmb{VHI} are used
by the now deprecated Y-type digital device, but are
ignored by the U device.  A warning message is emitted if any of these three parameters
are used in the model card for a U device.

Another caveat is that closely spaced input transitions to the \Xyce{} digital behavioral
models may not be accurately reflected in the output states.  In particular, input-state
changes spaced by more than \texttt{DELAY} seconds have independent effects on the output
states. However, two input-state changes (S1 and S2) that occur within \texttt{DELAY} seconds
(e.g., at time=t1 and time=t1+0.5*\texttt{DELAY}) have the effect of masking the effects
of S1 on the device's output states, and only the effects of S2 are propagated to the
device's output states.

\paragraph{DCOP Calculations for Flip-Flops and Latches}
The behavior of the digital devices during the DC Operating Point (DCOP) calculations
can be controlled via the \texttt{IC1} and \texttt{IC2} instance parameters and the
\texttt{DIGINITSTATE} device option.  See ~\ref{Options_Reference} for more details on the
syntax for device options.  Also, this section applies to the Y-Type Behavioral Digital Devices
discussed in ~\ref{YTypeDigitalDevice}.

The \texttt{IC1} instance parameter is supported for all gate types.  The \texttt{IC2}
instance parameter is supported for all gate types that have two outputs.  These instance
parameters allow the outputs of individual gates to be set to known states (either
\texttt{TRUE (1)} or \texttt{FALSE (0)}) during the DCOP calculation, irregardless of their 
input state(s).  There are two caveats.  First, the \texttt{IC1} and \texttt{IC2} settings 
at a given gate will override the global effects of the \texttt{DIGINITSTATE} option, 
discussed below, at that gate.  Second, \texttt{IC1} and \texttt{IC2} do not support the
\texttt{X}, or ``undetermined'', state discussed below.

The \texttt{DIGINITSTATE} option only applies to the DLTCH, DFF, JKFF and TFF devices.  It was added
for improved compatibility with PSpice.  It sets the initial state of all flip-flops and 
latches in the circuit: 0=clear, 1=set, 2=\texttt{X}.  At present, the use of the
\texttt{DIGINITSTATE} option during the DCOP is the only place that \Xyce{} supports the
\texttt{X}, or ``undetermined'', state.  The \texttt{X} state is modeled in \Xyce{} by having the
DLTCH, DFF, JKFF and TFF outputs simultaneously ``pulled-up'' and ``pulled-down''.  That approach typically 
produces an output level, for the \texttt{X} state, that is approximately halfway between the 
voltage levels for \texttt{TRUE} and \texttt{FALSE} (e.g., halfway between \texttt{V\_HI} and 
\texttt{V\_LO}). As mentioned above, the \texttt{IC1} and
\texttt{IC2} instance parameters take precedence at a given gate.

\Xyce{} also supports a default \texttt{DIGINITSTATE}, whose value is 3.  For this default value,
for the DFF, JKFF, TFF and DLTCH devices, \Xyce{} enforces $Q$ and $\bar{Q}$ being different at DCOP, if 
both \texttt{PREB} and \texttt{CLRB} are \texttt{TRUE} . The behavior of the DFF, JKFF and DLTCH 
devices at the DCOP for \texttt{DIGINITSTATE=3} is shown in Tables ~\ref{dffTruthTable},  ~\ref{jkffTruthTable}
and ~\ref{dltchTruthTable}.  In these three tables, the $X$ state denotes the ``Don't Care'' 
condition, where the input state can be 0, 1 or the ``undetermined'' state.
The first row in each truth-table (annotated with $*$) is ``unstable'', and will change to 
a state with $Q$ and $\bar{Q}$ being different once both \texttt{PREB} and \texttt{CLRB} are not 
both in the \texttt{FALSE} state.

The behavior of the TFF device at the DCOP, for the default \texttt{DIGINITSTATE} of 3, is
simpler, and is not shown as a table.  The design decision was to have $Q$ and $\bar{Q}$ be
different, with the $Q$ value equal to the state of the $T$ input.

%%
%% DFF Truth Table for DIGINITSTATE=3
%%
\LTXtable{\textwidth}{dffTruthTableTbl}

%%
%% DLTCH Truth Table for DIGINITSTATE=3
%%
\LTXtable{\textwidth}{dltchTruthTableTbl}

%%
%% JKFF Truth Table for DIGINITSTATE=3
%%
\LTXtable{\textwidth}{jkffTruthTableTbl}
