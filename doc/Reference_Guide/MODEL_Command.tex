% Sandia National Laboratories is a multimission laboratory managed and
% operated by National Technology & Engineering Solutions of Sandia, LLC, a
% wholly owned subsidiary of Honeywell International Inc., for the U.S.
% Department of Energy’s National Nuclear Security Administration under
% contract DE-NA0003525.

% Copyright 2002-2024 National Technology & Engineering Solutions of Sandia,
% LLC (NTESS).

\label{modelCommand}
The \texttt{.MODEL} command provides a set of device parameters to be
referenced  by device instances in the circuit.

\begin{Command}

\format
.MODEL <model name> <model type> (<name>=<value>)*

\examples
\begin{alltt}
.MODEL RMOD R (RSH=1)
.MODEL MOD1 NPN BF=50 VAF=50 IS=1.E-12 RB=100 CJC=.5PF TF=.6NS
.MODEL NFET NMOS(LEVEL=1 KP=0.5M VTO=2V)
\end{alltt}

\arguments

\begin{Arguments}

\argument{model name}
The model name used to reference the model.

\argument{model type}

The model type used to define the model.  This determines if the model
is (for example) a resistor, or a MOSFET, or a diode, etc.  For
transistors, there will usually be more than one type possible, such as
NPN and PNP for BJTs, and NMOS and PMOS for MOSFETs.

\argument{\vbox{\hbox{name\hfil}\hbox{value}}}

The name of a parameter and its value.  Most models will have a list of
parameters available for specification.  Those which are not set will
receive default values.  Most will be floating point numbers, but some
can be integers and some can be strings, depending on the definition of
the model.

\end{Arguments}

\comments
\index{\texttt{.MODEL}!subcircuit scoping}The scoping rules for models are:
\begin{XyceItemize}
\item If a \texttt{.MODEL}, statement is included in the main circuit 
netlist, then it is accessible from the main circuit and all subcircuits. 
\item \texttt{.MODEL} statements defined within a subcircuit are scoped 
to that subciruit definition.  So, their models are only accessible within 
that subcircuit definition, as well as within ``nested subcircuits'' also 
defined within that subcircuit definition.
\end{XyceItemize}

Additional illustative examples of scoping are given in the
``Working with Subcircuits and Models'' section of the \Xyce{} Users' 
Guide\UsersGuide. 

A model name can be the same as a device name in \Xyce{}.  However, that
usage will generate a warning message during netlist parsing.  
The reason is that it can lead to ambiguous \texttt{.PRINT} lines 
when a model parameter and instance parameter, for a given device, have
the same name but a different meaning.  For example, \texttt{R1} 
could be used as both a resistor device-name, and as a resistor model-name.
However, \texttt{.PRINT TRAN R1:R} would then be ambiguous.  In addition,
the use of duplicate model and device names is not recommended if those
names will be used within a \Xyce{} expression since that can result in 
an ambiguous expression.

\end{Command}

\subsubsection{\textrmb{LEVEL} Parameter}
\index{model!level parameter}
\index{level parameter}

A common parameter is the \textrmb{LEVEL} parameter, which is set to an
integer value.  This parameter will define exactly which model of the
given type is to be used.  For example, there are many different
available MOSFET models.  All of them will be specified using the same
possible names and types.  The way to differentiate (for example)
between the BSIM3 model and the PSP model is by setting the appropriate
\textrmb{LEVEL}.

\subsubsection{Model Binning}
\index{model!binning}
\index{model binning}

Model binning is supported for MOSFET models.  For model binning, the netlist 
contains a set of similar \texttt{.MODEL} cards which correspond to different 
sizing information (length and width).  They are similar in that they are for the same
model (and same \texttt{LEVEL} number), and have the same prefix.  They are different in that 
they have different \texttt{lmin,lmax,wmin,wmax} parameters, and the name suffix will be
the bin number.  For a MOSFET device instance, \Xyce{} will automatically select the
appropriate binned model, based on the \texttt{L} and \texttt{W} parameters of that 
instance.   It will only seach the models with matching name prefixes, and can only work if
all the binned models have specified all the \texttt{lmin,lmax,wmin,wmax} parameters.

Model binning is enabled by default.  To disable it, specify \texttt{.options parser model\_binning=false}.

\begin{figure}[htbp]
  \begin{centering}
    \shadowbox{
      \begin{minipage}{0.9\textwidth}
        \begin{vquote}
\color{blue}* Model binning example adapted from the BSIM4 tests\color{black}
m1 2 1 0 b nch L=0.11u W=10.1u NF=5 rgeomod=1 geomod=0
vgs 1 0 1.2
vds 2 0 1.2
Vb b 0 0.0

.dc vds 0.0 1.21 0.02 vgs 0.2 1.21 0.2

.print dc v(2) v(1) i(vds)

* model binning
.model nch.1 nmos(level=14 
+ lmin=0.1u lmax=20u 
+ wmin=0.1u wmax=10u)
.model nch.2 nmos(level=14 
+ lmin=0.1u lmax=20u 
+ wmin=10u  wmax=100u)

.end
\end{vquote}
\end{minipage}
}
\caption[Model Binning Example]
{Model Binning Example\label{binningExample} }
\end{centering}
\end{figure}

\subsubsection{Length Scaling}
\index{model!scale}
\index{scale}

It can be convenient to specify the lengths and widths for MOSFET instances in scaled units.  
To enable this, the netlist should include \texttt{.options parser scale} or \texttt{.option scale}.
This feature is only supported in MOSFET compact models.  An example usage is given in figure~\ref{scaleExample}.
In this example, the scaled length and width for transistor \texttt{mn1} is \texttt{l=5.0e-6} and \texttt{w=175e-6}.
\begin{figure}[htbp]
  \begin{centering}
    \shadowbox{
      \begin{minipage}{0.9\textwidth}
        \begin{vquote}
\color{blue}mos level 1 model cmos inverter using scale]\color{black}
.tran 20ns 6us
.print tran v(vout) v(in) v(1)
vdddev 	vdd	0	5v
rin	in	1	1k
vin1  1	0  5v pulse (5v 0v 1.5us 5ns 5ns 1.5us 3us)
r1    vout  0  10k  
c2    vout  0  0.1p 
mn1   vout  in 0 0 cd4012\_nmos l=5 w=175 
mp1   vout in vdd vdd cd4012\_pmos l=5 w=270 

.options parser scale=1.0e-6

\color{blue}* also valid:
*.option scale=1.0e-6\color{black}

.model cd4012\_pmos pmos (level=1 uo=310  vto=-1.6 tox=6e-08)
.model cd4012\_nmos nmos (level=1 uo=190 vto=1.679 tox=6e-08)
.end
\end{vquote}
\end{minipage}
}
\caption[Scale Example]
{Scale Example\label{scaleExample} }
\end{centering}
\end{figure}

